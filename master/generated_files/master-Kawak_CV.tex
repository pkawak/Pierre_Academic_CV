\begin{document}
\begin{center}
  {\LARGE \textbf{Pierre Kawak, Ph.D.} }\\[1ex]
  +1 (801) 762-7999 $\bullet$ \href{mailto:pskawak@gmail.com}{\tt pskawak@gmail.com} $\bullet$ \href{https://linktr.ee/pkawak}{\tt linktr.ee/pkawak}\\
\end{center}
\begin{tabitemize}
  \item Simulation engineer and physical systems modeler with 11+ years of experience developing high-performance modeling tools for polymers, composites, and crystallization.
  \item Proven track record building custom simulation engines in C++/CUDA, analyzing dynamical behavior, and driving insights from high-dimensional, terabyte-scale datasets.
  \item Thrive in small, interdisciplinary teams tackling difficult physical problems.
  \item Excited to extend skills to satellite systems and customer-impacting technologies.
\end{tabitemize}
\vspace{-1.4\baselineskip}
\section*{Technical Skills}
\begin{tabitemize}
  \item \textbf{Programming \& Software Engineering}: Python, C++, Bash, CUDA, Julia, R | Debugging (GDB, valgrind), Profiling (callgrind, nvprof) | Git, Unit Tests, Benchmarking
  \item \textbf{Modeling \& Systems Analysis}: Dynamical Systems, Statistical Mechanics, Free Energy Methods, Phase Transitions, Coarse-Grained \& Atomistic Systems
  \item \textbf{HPC \& Scalable Workflows}: Parallel Computing, Slurm, Open MPI, Workflow Automation, Cluster Management, 50+ TB Data Pipelines
  \item \textbf{Data Science \& Visualization}: NumPy, Pandas, Matplotlib, Scikit-learn, Bayesian Optimization, 3D Data Visualization (VMD, OVITO)
  \item \textbf{Communication \& Leadership}: Scientific Writing (5 publications), Public Speaking (27+ talks), Mentorship (15+ trainees), DEI Advocacy, Event Coordination
\end{tabitemize}
\vspace{-1.2\baselineskip}
\section*{Research Experience}
\vspace{-0.7\baselineskip}
\begin{longtable}{@{\extracolsep{\fill}}p{0.09\textwidth} p{0.37\textwidth} p{0.30\textwidth} r }
  \textbf{Postdoc} & \textbf{University of South Florida} & \textbf{Prof. David Simmons} & \textbf{2022 -- Present}\\
\end{longtable}
\vspace{-0.5\baselineskip}
\begin{tabitemize}
  \item Simulated polymer glassy dynamics, identifying sequences with enhanced thermal stibility \& guiding synthesis by collaborating with chemists.
  \item Spearheaded a 30$\times$ speed-up in polymer dynamics analysis via bottleneck resolution.
  \item Extended group's molecular simulation codebase, improving researcher onboarding.
  \item Automated large-scale HPC workflows across >50 TB of simulation data.
  \item Mentored 11 researchers in HPC, simulation, Git, and analysis, strengthening lab capabilities and documentation.
  \item Delivered 17+ conference talks \& won sci.~comm.~awards at GRC (2024) \& USF (2023).
\end{tabitemize}
\vspace{-1.8\baselineskip}
\begin{longtable}{@{\extracolsep{\fill}}p{0.09\textwidth} p{0.37\textwidth} p{0.30\textwidth} r }
  \textbf{Ph.D.} & \textbf{Brigham Young University} & \textbf{Prof. Douglas Tree} & \textbf{2017 -- 2022}\\
\end{longtable}
\vspace{-1.4\baselineskip}
\begin{tabitemize}
  \item Built 2 GPU-accelerated Monte Carlo engines (C++/CUDA) from first principles, enabling 100$\times$ faster crystallization studies \& published 3D free energy landscapes.
  \item Iteratively tested, benchmarked, and profiled algorithms to ensure accurate structure–property predictions and scalable performance.
  \item Designed order parameters to track molecular transitions \& structural reorganization.
  \item Analyzed large-scale trajectories to extract structural \& kinetic trends.
  \item Mentored 4 undergrads and co-authored 2 papers \& 6 conference abstracts.
  \item Played key role in a successful \$500K NSF CAREER grant.
  \item Earned APS Distinguished Student \& BYU Presentation Award for science comm.
\end{tabitemize}
\vspace{-0.8\baselineskip}
\begin{longtable}{@{\extracolsep{\fill}}p{0.09\textwidth} p{0.37\textwidth} p{0.30\textwidth} r }
  \textbf{M.S.} & \textbf{American University of Sharjah} & \textbf{Prof. Ghaleb Husseini} & \textbf{2015 -- 2017}\\
\end{longtable}
\vspace{-1.0\baselineskip}
\begin{tabitemize}
  \item Designed estrone-functionalized ultrasound-sensitive liposomes for breast cancer therapy, enhancing drug stability \& controlled release.
  \item Tuned release kinetics via experimental parameter sweeps \& signal optimization.
  \item Standardized lab protocols across teams, boosting reproducibility \& collaboration.
  \item Presented at 3 conferences; earned Best Talk Award at AUS Biomedi.~Eng.~Symposium.
\end{tabitemize}
\vspace{-2.3\baselineskip}
\section*{Leadership \& Community Engagement}
\begin{tabitemize}
  \item \textbf{President, Early Career Researchers in Polymer Physics (2022–):} Led a global 550-member community \& organized the 2023 Virtual Symposium with 150+ attendees.
  \item \textbf{President \& Founder, USF Postdoctoral Scholar Association (2023–):} Served 200+ postdocs via career events, DEI initiatives, \& the NPA-funded ELEVATE Talk Series.
  \item \textbf{President \& Founder, BYU Chem.~Eng.~Grad.~Student Council (2019–2022):} Directed recruitment, outreach, \& well-being programs impacting department policy.
\end{tabitemize}
\vspace{0.5\baselineskip}
\begin{document}
\begin{center}
  {\LARGE \textbf{Pierre Kawak, Ph.D.} }\\[1ex]
  +1 (801) 762-7999 $\bullet$ \href{mailto:pskawak@gmail.com}{\tt pskawak@gmail.com} $\bullet$ \href{https://linktr.ee/pkawak}{\tt linktr.ee/pkawak}\\
\end{center}
\begin{tabitemize}
  \item Data Science Lead with 11+ years of mentoring engineers, scientists, \& developers to deliver innovative computational solutions to complex material challenges.
  \item Proven success driving modeling projects in high-performance computing (HPC) \& cloud environments, including end-to-end software development in C++ \& Python.
  \item Strong foundation in engineering, mathematics, optimization, \& algorithm design, complemented by a consistent record of collaboration with experimentalists.
  \item Recognized for science communication through 5 publications \& 27+ conference talks.
  \item Passionate about leading ML \& simulation projects to transform engineering design.
\end{tabitemize}
\vspace{-1.7\baselineskip}
\section*{Technical Skills}
\begin{tabitemize}
  \item \textbf{Software Engineering:} Python, C++, C, CUDA, Julia, MATLAB, Bash, Git
  \item \textbf{Machine Learning \& Optimization:} Regression, Predictive Modeling, LLM Prompting, Time Series Analysis, Data Mining, Scikit-learn, Optimization Algorithms
  \item \textbf{High-Performance \& Cloud Computing:} Slurm, Open MPI, Workflow Automation, Large-Scale Data Processing (50TB+), Parallelization, Cloud \& HPC Pipelines
  \item \textbf{Data Analysis \& Visualization:} NumPy, Pandas, Matplotlib, OVITO, VMD, Advanced Scientific Visualization \& Illustration
  \item \textbf{Modeling \& Simulation:} Monte Carlo, Molecular Dynamics, LAMMPS, GROMACS, Gaussian, Coarse-Graining \& Atomistic Models
  \item \textbf{Communication \& Leadership:} Scientific Writing (5 publications), Conference Talks (27+), Technical Documentation, Mentoring (11 researchers), DEI Initiatives
\end{tabitemize}
\vspace{-1.7\baselineskip}
\section*{Research Experience}
\vspace{-0.8\baselineskip}
\begin{longtable}{@{\extracolsep{\fill}}p{0.09\textwidth} p{0.37\textwidth} p{0.30\textwidth} r }
  \textbf{Postdoc} & \textbf{University of South Florida} & \textbf{Prof. David Simmons} & \textbf{2022 -- Present}\\
\end{longtable}
\vspace{-1.5\baselineskip}
\begin{tabitemize}
  \item Led multi-variable modeling \& simulation campaigns on nanocomposites \& copolymers, optimizing materials design via high-dimensional parameter sweeps.
  \item Architected \& maintained HPC simulation \& analysis pipelines processing 50+ TB of data, reducing analysis time by 90\% \& earning NSF ACCESS compute resource award.
  \item Mentored \& trained 11 researchers in simulation methods, Git workflows, \& job automation, earning APS Career Mentor Fellowship.
  \item Created comprehensive technical documentation \& onboarding tutorials.
  \item Delivered 17 talks at conferences; recognized at USF (2023) \& GRC (2024).
\end{tabitemize}
\vspace{-1.0\baselineskip}
\begin{longtable}{@{\extracolsep{\fill}}p{0.09\textwidth} p{0.37\textwidth} p{0.30\textwidth} r }
  \textbf{Ph.D.} & \textbf{Brigham Young University} & \textbf{Prof. Douglas Tree} & \textbf{2017 -- 2022}\\
\end{longtable}
\vspace{-1.4\baselineskip}
\begin{tabitemize}
  \item Designed \& implemented two GPU-accelerated Monte Carlo modeling platforms (35K+ C++/CUDA lines each) to simulate crystallization \& accelerate discovery by 100$\times$.
  \item Engineered reusable simulation modules with object-oriented design, enabling flexible modeling of diverse chemical systems \& sampling strategies.
  \item Integrated unit testing infrastructure to ensure long-term reliability \& maintainability.
  \item Mentored 4 researchers, co-authoring 2 peer-reviewed publications \& 6 abstracts.
  \item Earned APS Distinguished Student Award (2022) \& BYU Presentation Award (2021) for scientific communication \& research excellence.
  \item Played key role in a successful \$500K NSF CAREER proposal.
\end{tabitemize}
\vspace{-0.7\baselineskip}
\begin{longtable}{@{\extracolsep{\fill}}p{0.09\textwidth} p{0.37\textwidth} p{0.30\textwidth} r }
  \textbf{M.S.} & \textbf{American University of Sharjah} & \textbf{Prof. Ghaleb Husseini} & \textbf{2015 -- 2017}\\
\end{longtable}
\vspace{-1.2\baselineskip}
\begin{tabitemize}
  \item Developed ultrasound-responsive drug delivery systems for breast cancer treatment, ensuring validity via NMR \& DLS.
  \item Analyzed dynamic release kinetics with ultrasound, tuning system parameters to improve clinical stability and delivery efficacy.
  \item Standardized lab protocols to boost reproducibility, collaboration, \& data integrity.
  \item Presented at 3 conferences, earning Best Talk Award at AUS Biomed.~Eng.~Symposium.
\end{tabitemize}
\vspace{-2.1\baselineskip}
\section*{Leadership \& Community Engagement}
\begin{tabitemize}
  \item \textbf{President, Early Career Researchers in Polymer Physics (2022–):} Led a global 550-member community \& organized the 2023 Virtual Symposium with 150+ attendees.
  \item \textbf{President \& Founder, USF Postdoctoral Scholar Association (2023–):} Served 200+ postdocs via career events, DEI initiatives, \& the NPA-funded ELEVATE Talk Series.
  \item \textbf{President \& Founder, BYU Chem.~Eng.~Grad.~Student Council (2019–2022):} Directed recruitment, outreach, \& well-being programs impacting department policy.
\end{tabitemize}
\begin{document}
\begin{center}
  {\LARGE \textbf{Pierre Kawak, Ph.D.} }\\[1ex]
  +1 (801) 762-7999 $\bullet$ \href{mailto:pskawak@gmail.com}{\tt pskawak@gmail.com} $\bullet$ \href{https://linktr.ee/pkawak}{\tt linktr.ee/pkawak}\\
\end{center}
\begin{tabitemize}
  \item Computational scientist with 11 years of experience designing atomistic simulations, automating high-throughput modeling workflows, and analyzing large structural datasets in materials and therapeutic systems.
  \item Skilled in Python, C, cloud computing, \& simulation such as LAMMPS \& GROMACS.
  \item Developed custom codebases and data pipelines (>50TB) to accelerate insights into copolymer design and molecular crystallization.
  \item Experience includes sequence-specific copolymer modeling for property optimization and initial cheminformatics tool development (e.g., RDKit).
  \item Passionate about applying AI/ML and molecular modeling to drug discovery.
\end{tabitemize}
\vspace{-1.4\baselineskip}
\section*{Technical Skills}
\begin{tabitemize}
  \item \textbf{Programming \& Software Development}: Python, R, C++, CUDA, Bash, GitHub, Jupyter, Linux (CLI), Object-Oriented Design, Unit Testing, Benchmarking
  \item \textbf{Molecular Modeling \& Simulation}: Gaussian, Molecular Dynamics (LAMMPS, GROMACS), Free Energy Calculations, Coarse-Graining, Atomistic Force Fields (AMBER, OPLS), Monte Carlo Methods, RDKit, Sequence-Structure-Property Relationships
  \item \textbf{Data Science \& Visualization}: Scikit-learn, Bayesian Optimization, NumPy, Pandas, Matplotlib, VMD, OVITO, High-Dimensional Large-Scale Data Analysis (>50TB)
  \item \textbf{HPC \& Cloud Infrastructure}: Slurm, Open MPI, Cluster Management, Parallelization
  \item \textbf{Communication \& Leadership}: Scientific Writing (5 publications), Public Speaking (27+ talks), Mentorship (15+ trainees), DEI Advocacy, Event Coordination
\end{tabitemize}
\vspace{-1.2\baselineskip}
\section*{Research Experience}
\vspace{-0.7\baselineskip}
\begin{longtable}{@{\extracolsep{\fill}}p{0.09\textwidth} p{0.37\textwidth} p{0.30\textwidth} r }
  \textbf{Postdoc} & \textbf{University of South Florida} & \textbf{Prof. David Simmons} & \textbf{2022 -- Present}\\
\end{longtable}
\vspace{-1.4\baselineskip}
\begin{tabitemize}
  \item Automated sequence-specific copolymer screening to identify formulations with elevated glass transition temperatures ($T_g$).
  \item Simulated deformation via molecular dynamics to create design rules for composites.
  \item Curated \& analyzed >50TB of simulation data using parallel Python/C++ pipelines, improving study turnaround time by 90\% and securing NSF HPC grant.
  \item Mentored 15+ researchers in simulation, Git workflows, \& cloud computing practices; selected for APS Career Mentorship Fellowship.
  \item Delivered 17+ conference talks \& won sci.~comm.~awards at GRC (2024) \& USF (2023).
\end{tabitemize}
\vspace{-1.8\baselineskip}
\begin{longtable}{@{\extracolsep{\fill}}p{0.09\textwidth} p{0.37\textwidth} p{0.30\textwidth} r }
  \textbf{Ph.D.} & \textbf{Brigham Young University} & \textbf{Prof. Douglas Tree} & \textbf{2017 -- 2022}\\
\end{longtable}
\vspace{-1.4\baselineskip}
\begin{tabitemize}
  \item Built two Monte Carlo simulation codes (C++/CUDA) from scratch, enabling 100$\times$ faster crystallization studies \& published 3D free energy landscapes.
  \item Iteratively tested, benchmarked, and profiled algorithms to ensure accurate structure–property predictions and scalable performance.
  \item Designed custom order parameters to analyze large 3D structural datasets, quantify molecular transitions and crystallization dynamics.
  \item Mentored 4 undergrads and co-authored 2 papers \& 6 conference abstracts.
  \item Played key role in a successful \$500K NSF CAREER grant.
  \item Earned APS Distinguished Student \& BYU Presentation Award for science comm.
\end{tabitemize}
\vspace{-0.4\baselineskip}
\begin{longtable}{@{\extracolsep{\fill}}p{0.09\textwidth} p{0.37\textwidth} p{0.30\textwidth} r }
  \textbf{M.S.} & \textbf{American University of Sharjah} & \textbf{Prof. Ghaleb Husseini} & \textbf{2015 -- 2017}\\
\end{longtable}
\vspace{-0.7\baselineskip}
\begin{tabitemize}
  \item Designed estrone-functionalized ultrasound-sensitive liposomes for breast cancer therapy, enhancing drug stability \& controlled release.
  \item Validated encapsulation \& release kinetics using DLS \& NMR, optimizing ultrasonic parameters for therapeutic viability.
  \item Standardized lab protocols across teams, boosting reproducibility \& collaboration.
  \item Presented at 3 conferences; earned Best Talk Award at AUS Biomedi.~Eng.~Symposium.
\end{tabitemize}
\vspace{-2.3\baselineskip}
\section*{Leadership \& Community Engagement}
\begin{tabitemize}
  \item \textbf{President, Early Career Researchers in Polymer Physics (2022–):} Led a global 550-member community \& organized the 2023 Virtual Symposium with 150+ attendees.
  \item \textbf{President \& Founder, USF Postdoctoral Scholar Association (2023–):} Served 200+ postdocs via career events, DEI initiatives, \& the NPA-funded ELEVATE Talk Series.
  \item \textbf{President \& Founder, BYU Chem.~Eng.~Grad.~Student Council (2019–2022):} Directed recruitment, outreach, \& well-being programs impacting department policy.
\end{tabitemize}
\vspace{0.5\baselineskip}
\begin{document}
\begin{center}
  {\LARGE \textbf{Pierre Kawak, Ph.D.} }\\[1ex]
  +1 (801) 762-7999 $\bullet$ \href{mailto:pskawak@gmail.com}{\tt pskawak@gmail.com} $\bullet$ \href{https://linktr.ee/pkawak}{\tt linktr.ee/pkawak}\\
\end{center}
\begin{tabitemize}
  \item Chemical engineer \& computational scientist with 11+ years of experience solving technical challenges in simulation, modeling, \& materials design.
  \item Expert in Microsoft Office, C++, Python, \& process modeling with a strong foundation in thermodynamics, transport, \& kinetics.
  \item Built custom simulation engines \& analysis pipelines (100$\times$ speedups), developed training content, \& mentored 15+ researchers.
  \item Recognized with fellowships \& awards for research, mentorship, \& communication
  \item Passionate about client support, technical instruction, \& scalable scientific software.
\end{tabitemize}
\vspace{-1.0\baselineskip}
\section*{Technical Skills}
\begin{tabitemize}
  \item \textbf{Programming Languages}: C, C++, CUDA, Python, C, MATLAB, R, Bash
  \item \textbf{Software Development}: Visual Studio, Git, GitHub, Linux CLI, Windows API
  \item \textbf{Modeling \& Simulation}: LAMMPS, GROMACS, Monte Carlo, Molecular Dynamics (MD), Gaussian, AMBER, OPLS
  \item \textbf{Data Analysis \& Visualization}: Scikit-learn, NumPy, Pandas, PyQt, Tkinter, Matplotlib, VMD, OVITO
  \item \textbf{High-Performance Computing}: Slurm, Open MPI, Parallelization, Multithreading, Workflow Automation, Cluster Management
  \item \textbf{Training \& Communication}: Technical Writing (5 publications), Scientific Presentations (27+), Mentoring (15+), Workshop Design \& Coordination
\end{tabitemize}
\vspace{-1.0\baselineskip}
\section*{Research Experience}
\vspace{-0.7\baselineskip}
\begin{longtable}{@{\extracolsep{\fill}}p{0.09\textwidth} p{0.37\textwidth} p{0.30\textwidth} r }
  \textbf{Postdoc} & \textbf{University of South Florida} & \textbf{Prof. David Simmons} & \textbf{2022 -- Present}\\
\end{longtable}
\vspace{-1.4\baselineskip}
\begin{tabitemize}
  \item Simulated deformation to enable predictive design of reinforced rubber composites.
  \item Built C++/Python tools to standardize analysis of rheology and chain dynamics.
  \item Automated HPC workflows to produce \& process 50+ TB of simulation data, reducing turnaround time by 90\%.
  \item Led structured mentorship for 11+ researchers, including workshops, code development, and onboarding documentation.
  \item Collaborated with polymer chemists to design \& evaluate copolymer sequences, guiding synthesis of thermally stable materials without feedstock or process changes.
  \item Presented research at 17+ venues; awarded at USF (2023) \& GRC (2024) meetings for advancements in rubber \& copolymer technology.
\end{tabitemize}
\vspace{-0.7\baselineskip}
\begin{longtable}{@{\extracolsep{\fill}}p{0.09\textwidth} p{0.37\textwidth} p{0.30\textwidth} r }
  \textbf{Ph.D.} & \textbf{Brigham Young University} & \textbf{Prof. Douglas Tree} & \textbf{2017 -- 2022}\\
\end{longtable}
\vspace{-0.7\baselineskip}
\begin{tabitemize}
  \item Developed 2 GPU-accelerated advanced-sampling Monte Carlo simulation engines, achieving 100$\times$ speedup in free energy calculations of polymer crystallization.
  \item Applied best practices including Git version control, automated testing, debugging, \& profiling to maintain robust simulation engines.
  \item Designed novel order parameters to quantify crystallization transitions in polymers.
  \item Mentored 4 undergraduates, coauthoring 2 publications \& 4 conference abstracts.
  \item Played key role in a successful \$500K NSF CAREER proposal.
\end{tabitemize}
\vspace{-0.4\baselineskip}
\begin{longtable}{@{\extracolsep{\fill}}p{0.09\textwidth} p{0.37\textwidth} p{0.30\textwidth} r }
  \textbf{M.S.} & \textbf{American University of Sharjah} & \textbf{Prof. Ghaleb Husseini} & \textbf{2015 -- 2017}\\
\end{longtable}
\vspace{-0.7\baselineskip}
\begin{tabitemize}
  \item Engineered ultrasound-sensitive estrone-targeted chemotherapeutic nanocarriers.
  \item Characterized release kinetics using DLS \& NMR, optimizing ultrasonic parameters for clinical stability \& efficacy.
  \item Standardized lab protocols across teams, boosting reproducibility \& collaboration.
  \item Presented at 3 conferences; earned Best Talk Award at AUS Biomed.~Eng.~Symposium.
\end{tabitemize}
\vspace{-2.3\baselineskip}
\section*{Leadership \& Community Engagement}
\begin{tabitemize}
  \item \textbf{President, Early Career Researchers in Polymer Physics (2022–):} Led a global 550-member community \& organized the 2023 Virtual Symposium with 150+ attendees.
  \item \textbf{President \& Founder, USF Postdoctoral Scholar Association (2023–):} Served 200+ postdocs via career events, DEI initiatives, \& the NPA-funded ELEVATE Talk Series.
  \item \textbf{President \& Founder, BYU Chem.~Eng.~Grad.~Student Council (2019–2022):} Directed recruitment, outreach, \& well-being programs impacting department policy.
\end{tabitemize}
\begin{document}
\begin{center}
  {\LARGE \textbf{Pierre Kawak, Ph.D.} }\\[1ex]
  +1 (801) 762-7999 $\bullet$ \href{mailto:pskawak@gmail.com}{\tt pskawak@gmail.com} $\bullet$ \href{https://linktr.ee/pkawak}{\tt linktr.ee/pkawak}\\
\end{center}
\vspace{-0.0cm}
Senior Scientist with 7+ years of experience applying molecular modeling, polymer physics, \& material characterization to advance next-generation material solutions.
Expertise includes free energy analysis, mechanical property prediction, polymer crystallization, \& multiscale simulations using molecular dynamics (MD), Monte Carlo (MC), \& CUDA-based models.
Proven ability to connect structure \& property through simulation \& experimental validation (e.g., $\bm{T_g}$ tuning, drug release kinetics, NMR/DLS).
Passionate about solving high-impact R\&D problems \& driving innovation in polymer technology.
\vspace{-1.0\baselineskip}
\section*{Research Experience}
\vspace{-0.5\baselineskip}
\begin{longtable}{@{\extracolsep{\fill}}p{0.34\textwidth} p{0.445\textwidth} r }
  \textbf{Postdoctoral Researcher} & \textbf{University of South Florida (USF)} & \textbf{2022 -- Present}\\
\end{longtable}
\vspace{-1.4\baselineskip}
\begin{tabitemize}
  \item Developed multiscale MD models using LAMMPS \& GROMACS to investigate stress relaxation \& nanoscale reinforcement in polymer composites, enabling predictive links between molecular structure \& mechanical performance.
  \item Created novel Python-based analysis methods for nonlinear rheology \& deformation behavior, supporting simulation-driven property optimization of filled rubber.
  \item Simulated copolymer thermal behavior using atomistic \& coarse-grained MD (OPLS), identifying high-$\bm{T_g}$ formulations with statistical validation without changing processing or feedstock chemistry, validating performance outcomes through structure-property modeling \& cross-functional collaboration with experimentalists.
  \item Optimized 50TB+ HPC workflows in bash/Python, improving simulation throughput by 90\% \& enabling rapid iteration across product development cycles; awarded NSF ACCESS Compute Resource Grant (2023).
  \item Mentored 11 junior researchers in HPC simulation \& data workflows, boosting collaboration, productivity, \& earning APS Career Mentor Fellowship (2023).
  \item Presented findings at 17 industrial, \& academic conferences, highlighting advancements in polymer characterization \& modeling \& earning poster awards at the Gordon Research Conference (2024) \& the USF Postdoc Research Symposium (2023).
\end{tabitemize}
\vspace{-0.7\baselineskip}
\begin{longtable}{@{\extracolsep{\fill}}p{0.34\textwidth} p{0.478\textwidth} r }
  \textbf{Doctoral Researcher} & \textbf{Brigham Young University} & \textbf{2017 -- 2022}\\
\end{longtable}
\vspace{-1.2\baselineskip}
\begin{tabitemize}
  \item Developed, optimized, \& deployed GPU-accelerated Monte Carlo simulations in C/CUDA, achieving $100\times$ speedup in property computes, enabling experimental comparison.
  \item Automated high-throughput simulations using Python, C++, MATLAB, \& R, reducing statistically significant paramter-sweep runtimes for data-driven characterization.
  \item Constructed first-ever 3D free energy landscapes to map ordering pathways \& phase behavior, offering first-principles insights into molecular crystallization processes.
  \item Created advanced phase diagrams using custom order parameters to quantify phase transitions \& molecular ordering via design of experiments, delivering new crystallization metrology tools.
  \item Visualized \& analyzed large datasets of 3D molecular configurations using VMD \& OVITO, extracting key structural \& kinetic insights.
  \item Mentored 4 undergraduate researchers, co-authoring 2 journal articles \& 6 conference abstracts, thereby supporting their transition to graduate-level careers.
  \item Secured research awards, e.g., APS Forum on Intl.~Physics Distinguished Student Award (2022) \& BYU Grad.~Student Society Professional Presentation Award (2021).
  \item Contributed critical preliminary findings that supported the successful NSF CAREER Award (\$500,000) proposal for continued crystallization research.
\end{tabitemize}
\vspace{-0.7\baselineskip}
\begin{longtable}{@{\extracolsep{\fill}}p{0.34\textwidth} p{0.48\textwidth} r }
  \textbf{Masters Researcher} & \textbf{American University of Sharjah} & \textbf{2015 -- 2017}\\
\end{longtable}
\vspace{-1.2\baselineskip}
\begin{tabitemize}
  \item Designed ultrasound-sensitive drug delivery systems to improve chemotherapy efficiency, winning the AUS Biomedical Engineering Symposium Best Talk Award (2016).
  \item Engineered estrone-functionalized liposomal drug carriers, enhancing breast cancer drug delivery precision.
  \item Formulated \& optimized self-assembling chemotherapy nanoparticles using the dry film method, enhancing drug stability \& controlled release kinetics.
  \item Validated encapsulation efficiency \& nanoparticle integrity through NMR \& DLS assays, ensuring high drug-loading capacity \& structural robustness.
  \item Quantified ultrasound-triggered drug release across frequency \& intensity gradients, identifying optimal acoustic parameters for clinical performance \& stability.
  \item Standardized lab protocols to improve reproducibility, collaboration, \& data integrity, increasing research efficiency across teams.
  \item Published findings in a peer-reviewed journal \& presented at 3 conferences, disseminating contributions to biomedical engineering \& drug delivery research.
\end{tabitemize}
\vspace{-1.5\baselineskip}
\section*{Leadership \& Community Engagement}
\vspace{-0.5\baselineskip}
\begin{longtable}{@{\extracolsep{\fill}}p{0.20\textwidth} p{0.585\textwidth} r }
  \textbf{President} & \textbf{Early Career Researchers in Polymer Physics} & \textbf{2022 -- Present}\\
\end{longtable}
\vspace{-1.2\baselineskip}
\begin{tabitemize}
  \item Led a 550-member global Slack community, organizing networking, technical, self-development, \& conference prep events, improving belonging of polymer researchers.
  \item Organized the 2023 Virtual Polymer Physics Symposium, a 2-day intl.~event with 150+ attendees, 4 technical sessions, a DEI discussion, \& a diverse career panel.% Prioritized financially constrained \& intl.~speakers, ensuring equitable research exposure.
\end{tabitemize}
\vspace{-0.7\baselineskip}
\begin{longtable}{@{\extracolsep{\fill}}p{0.28\textwidth} p{0.505\textwidth} r }
  \textbf{President and Founder} & \textbf{USF Postdoctoral Scholar Association} & \textbf{2023 -- Present}\\
\end{longtable}
\vspace{-1.2\baselineskip}
\begin{tabitemize}
  \item Served 200+ postdocs through career programming, networking events, \& advocacy, e.g., ELEVATE Talk Series, funded by NPA IMPACT Fellowship (2023, 6\% acc.~rate).
\end{tabitemize}
\vspace{-0.7\baselineskip}
\begin{longtable}{@{\extracolsep{\fill}}p{0.28\textwidth} p{0.538\textwidth} r }
  \textbf{President and Founder} & \textbf{BYU Chem.~Eng.~Graduate Student Council} & \textbf{2019 -- 2022}\\
\end{longtable}
\vspace{-1.2\baselineskip}
\begin{tabitemize}
  \item Organized dept. recruitment, social \& outreach events, social content, \& financial well-being initiatives, e.g., Recruitment Poster Event (2019–2021) \& BBQ Socials (2018–2021).
  \item Administered a financial health survey to assess graduate student well-being, influencing department policy discussions (2021).
\end{tabitemize}
\vspace{0.2\baselineskip}
\begin{document}
\begin{center}
  {\LARGE \textbf{Pierre Kawak, Ph.D.} }\\[1ex]
  $\bullet$ (801) 762-7999 $\bullet$ pskawak@gmail.com $\bullet$ linktr.ee/pkawak $\bullet$ \\
\end{center}
\vspace{-0.25cm}
Computational Modeling Researcher with 7+ years of experience in developing \& deploying first-principles \& molecular simulations to investigate polymer thermal \& mechanical properties, drug delivery systems, \& crystallization.
Career highlights include leading large-scale modeling campaigns using LAMMPS, GROMACS, AMBER, \& Python; optimizing data workflows to process 50TB+ datasets; \& mentoring 11 junior researchers in HPC methods.
Seeking to leverage deep computational expertise \& team leadership to drive modeling innovation \& reactor/process development at ExxonMobil.
\vspace{-1.2\baselineskip}
\section*{Research Experience}
\vspace{-0.9\baselineskip}
\begin{longtable}{@{\extracolsep{\fill}}p{0.34\textwidth} p{0.445\textwidth} r }
  \textbf{Postdoctoral Researcher} & \textbf{University of South Florida (USF)} & \textbf{2022 -- Present}\\
\end{longtable}
\vspace{-1.6\baselineskip}
\begin{tabitemize}
  \item Developed \& applied first-principles molecular dynamics (MD) simulations using LAMMPS \& GROMACS to probe stress relaxation \& deformation mechanics in polymer composites, yielding physical predictions to guide macroscopic performance.
  \item Designed custom Python-based analysis frameworks to extract nonlinear rheological behavior from simulation data, linking simulation outputs to phenomenological insights relevant for advanced rubber design \& performance/process optimization.
  \item Simulated coarse-grained \& atomistic copolymer sequences using OPLS \& multi-scale MD approaches to identify $\bm{T_g}$-enhancing formulations, enabling improved thermal properties without feedstock or process changes.
  \item Architected HPC workflows using bash \& Python to process >50TB of simulation data, reducing data pipeline runtime by 90\% \& enabling real-time iteration across modeling efforts; awarded NSF Discover ACCESS Compute Grant (2023).
  \item Mentored \& trained 11 researchers in HPC-enabled simulation, version control (Git), \& data management practices, strengthening team modeling capacity \& earning the APS Career Mentor Fellowship (2023).
  \item Presented modeling innovations \& materials simulation \& theory advancements at 17+ academic \& industry venues, earning Outstanding Poster honors at Gordon Research Conference (2024) \& USF Postdoctoral Symposium (2023) for contributions to predictive polymer performance modeling.
\end{tabitemize}
\vspace{-1.4\baselineskip}
\begin{longtable}{@{\extracolsep{\fill}}p{0.34\textwidth} p{0.478\textwidth} r }
  \textbf{Doctoral Researcher} & \textbf{Brigham Young University} & \textbf{2017 -- 2022}\\
\end{longtable}
\vspace{-1.6\baselineskip}
\begin{tabitemize}
  \item Developed, optimized, \& deployed GPU-accelerated Monte Carlo simulations in C/CUDA to investigate polymer crystallization thermodynamics, achieving $100\times$ speedup in property computes, enabling experimental comparison.
  \item Automated high-throughput, multi-variate simulations using Python, C++, bash, MATLAB, \& R to explore polymer morphology \& crystallization kinetics, reducing paramter-sweep runtimes \& enabling large-scale polymer crystal studies.%sweeping multi-dimensional parameters \& accelerating studies of polymer crystals.
  \item Constructed first-ever 3D free energy landscapes for polymer crystallization, differentiating metastable states \& ordering pathways inaccessible to classical methods.% \& resolving a long-standing theoretical controversy on polymer crystallization pathways.
  \item Derived advanced phase diagrams \& applied custom order parameters to quantify crystalline \& orientational order, classifying phase transitions in complex landscapes.
  \item Visualized molecular datasets in OVITO, extracting key structural \& kinetic insights.
  \item Mentored 4 undergraduate researchers, co-authoring 2 journal articles \& 6 conference abstracts, thereby supporting their transition to graduate-level careers.
  \item Secured research awards, e.g., APS Forum on Intl.~Physics Distinguished Student Award (2022) \& BYU Grad.~Student Society Professional Presentation Award (2021).
  \item Contributed critical preliminary findings that supported the successful NSF CAREER Award (\$500,000) proposal for continued crystallization research.
\end{tabitemize}
\vspace{-0.7\baselineskip}
\begin{longtable}{@{\extracolsep{\fill}}p{0.34\textwidth} p{0.48\textwidth} r }
  \textbf{Masters Researcher} & \textbf{American University of Sharjah} & \textbf{2015 -- 2017}\\
\end{longtable}
\vspace{-1.2\baselineskip}
\begin{tabitemize}
  \item Developed ultrasound-responsive drug delivery systems for targeted chemotherapy, integrating mechanistic understanding of acoustic propagation with kinetic drug release modeling; awarded Best Talk at AUS Biomedical Engineering Symposium (2016).
  \item Designed \& synthesized estrone-functionalized phospholipid liposomes to improve tumor-targeting precision, enhancing selective drug uptake in breast cancer treatment.% without sacrificing drug release.
  \item Formulated \& optimized self-assembling chemotherapy nanoparticles using the dry film method, enhancing drug stability \& controlled release kinetics.
  \item Validated encapsulation efficiency \& nanoparticle integrity through NMR \& DLS, ensuring high drug-loading capacity \& structural robustness.
  \item Quantified \& optimized ultrasound-triggered drug release kinetics, identifying ideal acoustic frequency \& intensity parameters to maximize on-demand release profiles--bridging lab-scale kinetics to potential clinical translation.
  \item Standardized cross-lab protocols \& workflows to improve reproducibility, collaboration, \& data integrity across research units.
  \item Disseminated findings in a peer-reviewed publication \& 3 conference presentations, contributing novel methods to the field of responsive drug delivery.
  \item Served as Instructor of Record for a Principles of Chemical Engineering recitation, teaching unit operation modeling in ASPEN HYSYS (e.g., reactors, pumps, separators); redesigned syllabus to emphasize industry-relevant software features.
\end{tabitemize}
\vspace{-1.6\baselineskip}
\section*{Leadership \& Community Engagement}
\vspace{-0.7\baselineskip}
\begin{longtable}{@{\extracolsep{\fill}}p{0.20\textwidth} p{0.585\textwidth} r }
  \textbf{President} & \textbf{Early Career Researchers in Polymer Physics} & \textbf{2022 -- Present}\\
\end{longtable}
\vspace{-1.0\baselineskip}
\begin{tabitemize}
  \item Led a 550-member global Slack community, organizing networking, technical, self-development, \& conference prep events, improving belonging of polymer researchers.
  \item Directed the 2023 Virtual Polymer Physics Symposium, a 2-day intl.~event with 150+ attendees, 4 technical sessions, a DEI discussion, \& a diverse career panel.% Prioritized financially constrained \& intl.~speakers, ensuring equitable research exposure.
\end{tabitemize}
\vspace{-0.5\baselineskip}
\begin{longtable}{@{\extracolsep{\fill}}p{0.28\textwidth} p{0.505\textwidth} r }
  \textbf{President and Founder} & \textbf{USF Postdoctoral Scholar Association} & \textbf{2023 -- Present}\\
\end{longtable}
\vspace{-1.0\baselineskip}
\begin{tabitemize}
  \item Served 200+ postdocs through career programming, networking events, \& advocacy, e.g., ELEVATE Talk Series, funded by NPA IMPACT Fellowship (2023, 6\% acc.~rate).
\end{tabitemize}
\vspace{-0.5\baselineskip}
\begin{longtable}{@{\extracolsep{\fill}}p{0.28\textwidth} p{0.538\textwidth} r }
  \textbf{President and Founder} & \textbf{BYU Chem.~Eng.~Graduate Student Council} & \textbf{2019 -- 2022}\\
\end{longtable}
\vspace{-1.0\baselineskip}
\begin{tabitemize}
  \item Organized dept. recruitment, social \& outreach events, social content, \& financial well-being initiatives, e.g., Recruitment Poster Event (2019–2021) \& BBQ Socials (2018–2021).
  \item Designed \& administered a financial health survey to assess graduate student well-being, influencing department policy discussions (2021).
\end{tabitemize}
\begin{document}
\begin{center}
  {\LARGE \textbf{Pierre Kawak, Ph.D.} }\\[1ex]
  $\bullet$ (801) 762-7999 $\bullet$ pskawak@gmail.com $\bullet$ linktr.ee/pkawak $\bullet$ \\
\end{center}
\vspace{-0.25cm}
Computational Modeling Researcher with 7+ years of experience in developing \& deploying first-principles \& molecular simulations to investigate polymer thermal \& mechanical properties, drug delivery systems, \& crystallization.
Career highlights include leading large-scale modeling campaigns using LAMMPS, GROMACS, AMBER, \& Python; optimizing data workflows to process 50TB+ datasets; \& mentoring 11 junior researchers in HPC methods.
Seeking to leverage deep computational expertise \& team leadership to drive modeling innovation \& reactor/process development at ExxonMobil.
\vspace{-1.2\baselineskip}
\section*{Research Experience}
\vspace{-0.9\baselineskip}
\begin{longtable}{@{\extracolsep{\fill}}p{0.34\textwidth} p{0.445\textwidth} r }
  \textbf{Postdoctoral Researcher} & \textbf{University of South Florida (USF)} & \textbf{2022 -- Present}\\
\end{longtable}
\vspace{-1.6\baselineskip}
\begin{tabitemize}
  \item Developed \& applied first-principles molecular dynamics (MD) simulations using LAMMPS \& GROMACS to probe stress relaxation \& deformation mechanics in polymer composites, yielding physical predictions to guide macroscopic performance.
  \item Designed custom Python-based analysis frameworks to extract nonlinear rheological behavior from simulation data, linking simulation outputs to phenomenological insights relevant for advanced rubber design \& performance/process optimization.
  \item Simulated coarse-grained \& atomistic copolymer sequences using OPLS \& multi-scale MD approaches to identify $\bm{T_g}$-enhancing formulations, enabling improved thermal properties without feedstock or process changes.
  \item Architected HPC workflows using bash \& Python to process >50TB of simulation data, reducing data pipeline runtime by 90\% \& enabling real-time iteration across modeling efforts; awarded NSF Discover ACCESS Compute Grant (2023).
  \item Mentored \& trained 11 researchers in HPC-enabled simulation, version control (Git), \& data management practices, strengthening team modeling capacity \& earning the APS Career Mentor Fellowship (2023).
  \item Presented modeling innovations \& materials simulation \& theory advancements at 17+ academic \& industry venues, earning Outstanding Poster honors at Gordon Research Conference (2024) \& USF Postdoctoral Symposium (2023) for contributions to predictive polymer performance modeling.
\end{tabitemize}
\vspace{-1.4\baselineskip}
\begin{longtable}{@{\extracolsep{\fill}}p{0.34\textwidth} p{0.478\textwidth} r }
  \textbf{Doctoral Researcher} & \textbf{Brigham Young University} & \textbf{2017 -- 2022}\\
\end{longtable}
\vspace{-1.6\baselineskip}
\begin{tabitemize}
  \item Developed, optimized, \& deployed GPU-accelerated Monte Carlo simulations in C/CUDA to investigate polymer crystallization thermodynamics, achieving $100\times$ speedup in property computes, enabling experimental comparison.
  \item Automated high-throughput, multi-variate simulations using Python, C++, bash, MATLAB, \& R to explore polymer morphology \& crystallization kinetics, reducing paramter-sweep runtimes \& enabling large-scale polymer crystal studies.%sweeping multi-dimensional parameters \& accelerating studies of polymer crystals.
  \item Constructed first-ever 3D free energy landscapes for polymer crystallization, differentiating metastable states \& ordering pathways inaccessible to classical methods.% \& resolving a long-standing theoretical controversy on polymer crystallization pathways.
  \item Derived advanced phase diagrams \& applied custom order parameters to quantify crystalline \& orientational order, classifying phase transitions in complex landscapes.
  \item Visualized molecular datasets in OVITO, extracting key structural \& kinetic insights.
  \item Mentored 4 undergraduate researchers, co-authoring 2 journal articles \& 6 conference abstracts, thereby supporting their transition to graduate-level careers.
  \item Secured research awards, e.g., APS Forum on Intl.~Physics Distinguished Student Award (2022) \& BYU Grad.~Student Society Professional Presentation Award (2021).
  \item Contributed critical preliminary findings that supported the successful NSF CAREER Award (\$500,000) proposal for continued crystallization research.
\end{tabitemize}
\vspace{-0.7\baselineskip}
\begin{longtable}{@{\extracolsep{\fill}}p{0.34\textwidth} p{0.48\textwidth} r }
  \textbf{Masters Researcher} & \textbf{American University of Sharjah} & \textbf{2015 -- 2017}\\
\end{longtable}
\vspace{-1.2\baselineskip}
\begin{tabitemize}
  \item Developed ultrasound-responsive drug delivery systems for targeted chemotherapy, integrating mechanistic understanding of acoustic propagation with kinetic drug release modeling; awarded Best Talk at AUS Biomedical Engineering Symposium (2016).
  \item Designed \& synthesized estrone-functionalized phospholipid liposomes to improve tumor-targeting precision, enhancing selective drug uptake in breast cancer treatment.% without sacrificing drug release.
  \item Formulated \& optimized self-assembling chemotherapy nanoparticles using the dry film method, enhancing drug stability \& controlled release kinetics.
  \item Validated encapsulation efficiency \& nanoparticle integrity through NMR \& DLS, ensuring high drug-loading capacity \& structural robustness.
  \item Quantified \& optimized ultrasound-triggered drug release kinetics, identifying ideal acoustic frequency \& intensity parameters to maximize on-demand release profiles--bridging lab-scale kinetics to potential clinical translation.
  \item Standardized cross-lab protocols \& workflows to improve reproducibility, collaboration, \& data integrity across research units.
  \item Disseminated findings in a peer-reviewed publication \& 3 conference presentations, contributing novel methods to the field of responsive drug delivery.
  \item Served as Instructor of Record for a Principles of Chemical Engineering recitation, teaching unit operation modeling in ASPEN HYSYS (e.g., reactors, pumps, separators); redesigned syllabus to emphasize industry-relevant software features.
\end{tabitemize}
\vspace{-1.6\baselineskip}
\section*{Leadership \& Community Engagement}
\vspace{-0.7\baselineskip}
\begin{longtable}{@{\extracolsep{\fill}}p{0.20\textwidth} p{0.585\textwidth} r }
  \textbf{President} & \textbf{Early Career Researchers in Polymer Physics} & \textbf{2022 -- Present}\\
\end{longtable}
\vspace{-1.0\baselineskip}
\begin{tabitemize}
  \item Led a 550-member global Slack community, organizing networking, technical, self-development, \& conference prep events, improving belonging of polymer researchers.
  \item Directed the 2023 Virtual Polymer Physics Symposium, a 2-day intl.~event with 150+ attendees, 4 technical sessions, a DEI discussion, \& a diverse career panel.% Prioritized financially constrained \& intl.~speakers, ensuring equitable research exposure.
\end{tabitemize}
\vspace{-0.5\baselineskip}
\begin{longtable}{@{\extracolsep{\fill}}p{0.28\textwidth} p{0.505\textwidth} r }
  \textbf{President and Founder} & \textbf{USF Postdoctoral Scholar Association} & \textbf{2023 -- Present}\\
\end{longtable}
\vspace{-1.0\baselineskip}
\begin{tabitemize}
  \item Served 200+ postdocs through career programming, networking events, \& advocacy, e.g., ELEVATE Talk Series, funded by NPA IMPACT Fellowship (2023, 6\% acc.~rate).
\end{tabitemize}
\vspace{-0.5\baselineskip}
\begin{longtable}{@{\extracolsep{\fill}}p{0.28\textwidth} p{0.538\textwidth} r }
  \textbf{President and Founder} & \textbf{BYU Chem.~Eng.~Graduate Student Council} & \textbf{2019 -- 2022}\\
\end{longtable}
\vspace{-1.0\baselineskip}
\begin{tabitemize}
  \item Organized dept. recruitment, social \& outreach events, social content, \& financial well-being initiatives, e.g., Recruitment Poster Event (2019–2021) \& BBQ Socials (2018–2021).
  \item Designed \& administered a financial health survey to assess graduate student well-being, influencing department policy discussions (2021).
\end{tabitemize}
\begin{document}
\begin{center}
  {\LARGE \textbf{Pierre Kawak, Ph.D.} }\\[1ex]
  $\bullet$ (801) 762-7999 $\bullet$ pskawak@gmail.com $\bullet$ linktr.ee/pkawak $\bullet$ \\
\end{center}
\vspace{-0.6cm}
\begin{tabitemize}
  \item Quantitative researcher with expertise in mathematical modeling, high-performance computing (HPC), and large-scale data analysis. 
  \item Skilled in Python, C++, and CUDA for developing statistical models, optimization techniques, and machine learning algorithms to solve complex real-world problems.
  \item Experienced in analyzing large datasets, identifying predictive patterns, and optimizing computational workflows.
  \item 7 years of computational expertise and 4 years of experimental expertise doing impactful research in molecular modeling and drug delivery.
  \item Authored $5$ peer-reviewed articles, contributing to advancements in copolymer theory, polymer dynamics modeling, filled rubber mechanics, \& cancer drug delivery.
  \item Presented at $27$ institutional, national, \& intl. conferences (e.g., APS, ACS, AIChE, USF, AUS, GRC, IoP, etc.) to diverse audiences from industry, govt., \& academia.
  \item Passionate about leveraging quantitative techniques for data-driven decision-making in financial markets.
\end{tabitemize}
\vspace{-2.0\baselineskip}
\section*{Research Experience}
\vspace{-0.9\baselineskip}
\begin{longtable}{@{\extracolsep{\fill}}p{0.34\textwidth} p{0.445\textwidth} r }
  \textbf{Postdoctoral Researcher} & \textbf{University of South Florida (USF)} & \textbf{2022 -- Present}\\
\end{longtable}
\vspace{-1.6\baselineskip}
\begin{tabitemize}
  \item Developed and implemented large-scale stochastic simulations using Python, C++, and CUDA to model complex system dynamics, securing an NSF Discover ACCESS Compute Resource Grant (2023) and leveraging high-performance computing (HPC) to process multi-terabyte datasets efficiently.
  \item Designed statistical models and optimization algorithms to analyze nonlinear system behaviors and high-dimensional parameter estimation, accelerating simulations by 100$\times$ through parallel computing.
  \item Applied machine learning techniques to extract predictive insights from complex datasets, improving model performance and decision-making.
  \item Built and automated data analysis pipelines using Python, C++, bash, and R, streamlining large-scale computational workflows.
  \item Mentored \& trained 11 researchers in HPC, version control, \& algorithm development, boosting collaboration, productivity, \& earning APS Career Mentor Fellowship (2023).
  \item Presented findings at 17 institutional, industrial, \& academic conferences, highlighting advancements in rubber \& copolymer technology, as well as polymer theory, \& earning the Outstanding Poster Award at the Gordon Research Conference (2024) \& the USF Annual Postdoctoral Research Symposium Best Poster Award (2023).
\end{tabitemize}
\vspace{-1.4\baselineskip}
\begin{longtable}{@{\extracolsep{\fill}}p{0.34\textwidth} p{0.478\textwidth} r }
  \textbf{Doctoral Researcher} & \textbf{Brigham Young University} & \textbf{2017 -- 2022}\\
\end{longtable}
\vspace{-1.6\baselineskip}
\begin{tabitemize}
  \item Developed and optimized GPU-accelerated Monte Carlo simulations in C++/CUDA, achieving a 100$\times$ speedup in large-scale probabilistic modeling and enabling efficient high-dimensional parameter estimation.
  \item Automated high-throughput simulations using Python, C++, bash, MATLAB, \& R, sweeping multi-dimensional parameters \& accelerating property predictions.
  \item Constructed the first-ever 3D free energy landscapes for polymer crystallization, differentiating order-formation pathways inaccessible to classical simulations.% \& resolving a long-standing theoretical controversy on polymer crystallization pathways.
  \item Visualized \& analyzed large datasets of 3D molecular configurations using VMD \& OVITO, extracting key structural \& kinetic insights.
  \item Wrote 2 journal articles with 2 mentored undergraduates, supporting their careers.
  \item Secured research awards, e.g., APS Forum on Intl.~Physics Distinguished Student Award (2022) \& BYU Grad.~Student Society Professional Presentation Award (2021).
  \item Presented at 6 conferences \& directly contributed to an NSF CAREER Award (\$500,000) for continued crystallization research by producing critical preliminary findings.
\end{tabitemize}
\vspace{-0.7\baselineskip}
\begin{longtable}{@{\extracolsep{\fill}}p{0.34\textwidth} p{0.48\textwidth} r }
  \textbf{Masters Researcher} & \textbf{American University of Sharjah} & \textbf{2015 -- 2017}\\
\end{longtable}
\vspace{-1.0\baselineskip}
\begin{tabitemize}
  \item Developed ultrasound-sensitive drug delivery systems to improve chemotherapy efficiency, winning the AUS Biomedical Engineering Symposium Best Talk Award (2016).
  \item Designed \& synthesized tumor-targeting liposomal drug carriers by functionalizing phospholipids with estrone ligands, improving breast cancer drug delivery precision.
  \item Formulated \& optimized self-assembling chemotherapy nanoparticles using the dry film method, enhancing drug stability \& controlled release kinetics.
  \item Validated encapsulation efficiency \& nanoparticle integrity through NMR \& DLS assays, ensuring high drug-loading capacity \& structural robustness.
  \item Characterized \& optimized ultrasound-triggered drug release kinetics, determining the ideal ultrasound frequency \& intensity for future clinical applications.
  \item Standardized lab protocols to improve reproducibility, collaboration, \& data integrity, increasing research efficiency across teams.
  \item Published findings in a peer-reviewed journal \& presented at 3 conferences, disseminating contributions to biomedical engineering \& drug delivery research.
\end{tabitemize}
\vspace{-1.6\baselineskip}
\section*{Leadership \& Community Engagement}
\vspace{-0.7\baselineskip}
\begin{longtable}{@{\extracolsep{\fill}}p{0.20\textwidth} p{0.585\textwidth} r }
  \textbf{President} & \textbf{Early Career Researchers in Polymer Physics} & \textbf{2022 -- Present}\\
\end{longtable}
\vspace{-1.0\baselineskip}
\begin{tabitemize}
  \item Led a 550-member global Slack community, organizing networking, technical, self-development, \& conference prep events, improving belonging of polymer researchers.
  \item Organized the 2023 Virtual Polymer Physics Symposium, a 2-day intl.~event with 150+ attendees, 4 technical sessions, a DEI discussion, \& a diverse career panel.% Prioritized financially constrained \& intl.~speakers, ensuring equitable research exposure.
\end{tabitemize}
\vspace{-0.5\baselineskip}
\begin{longtable}{@{\extracolsep{\fill}}p{0.28\textwidth} p{0.505\textwidth} r }
  \textbf{President and Founder} & \textbf{USF Postdoctoral Scholar Association} & \textbf{2023 -- Present}\\
\end{longtable}
\vspace{-1.0\baselineskip}
\begin{tabitemize}
  \item Served 200+ postdocs through career programming, networking events, \& advocacy, e.g., ELEVATE Talk Series, funded by NPA IMPACT Fellowship (2023, 6\% acc.~rate).
\end{tabitemize}
\vspace{-0.5\baselineskip}
\begin{longtable}{@{\extracolsep{\fill}}p{0.28\textwidth} p{0.538\textwidth} r }
  \textbf{President and Founder} & \textbf{BYU Chem.~Eng.~Graduate Student Council} & \textbf{2019 -- 2022}\\
\end{longtable}
\vspace{-1.0\baselineskip}
\begin{tabitemize}
  \item Organized dept. recruitment, social \& outreach events, social content, \& financial well-being initiatives, e.g., Recruitment Poster Event (2019–2021) \& BBQ Socials (2018–2021).
  \item Administered a financial health survey to assess graduate student well-being, influencing department policy discussions (2021).
\end{tabitemize}
\begin{document}
\begin{center}
  {\LARGE \textbf{Pierre Kawak, Ph.D.} }\\[1ex]
  +1 (801) 762-7999 $\bullet$ \href{mailto:pskawak@gmail.com}{\tt pskawak@gmail.com} $\bullet$ \href{https://linktr.ee/pkawak}{\tt linktr.ee/pkawak}\\
\end{center}
\begin{tabitemize}
  \item Computational materials scientist with 7+ years of developing physics-based \& data-driven models for solid mechanics, material deformation, \& nonlinear rheology.
  \item Specialized in thermo-elastoplastic modeling of polymeric materials using HPC, GPU-accelerated simulations, \& molecular dynamics frameworks, with direct applications in additive manufacturing \& process optimization.
  \item Passionate about advancing Freeform's mission via first-principles thinking, cross-disciplinary collaboration, \& creative problem-solving in automation technologies.
\end{tabitemize}
\vspace{-1.0\baselineskip}
\section*{Technical Skills}
\begin{tabitemize}
  \item \textbf{Modeling \& Simulation}: Molecular Dynamics (LAMMPS, GROMACS), Monte Carlo Sampling, Thermo-elastoplasticity, Deformation Modeling, Nonlinear Rheology
  \item \textbf{Programming \& Software Development}: Python, C++, CUDA, Bash, MATLAB, R, Modular Code Design, Unit Testing, Git
  \item \textbf{High-Performance Computing (HPC)}: Slurm, Open MPI, Parallel Computing, Workflow Automation, Real-Time Simulation Optimization, 50TB+ Datasets
  \item \textbf{Data Analysis \& Visualization}: scikit-learn, Bayesian Optimization, NumPy, Pandas, Matplotlib, VMD, OVITO, 3D Visualization, Simulation Output Parsing
  \item \textbf{Relevant Domains}: Computational Solid Mechanics, Polymer Physics, Copolymers, Nanocomposites, Glass Transition, Material Degradation
  \item \textbf{Collaboration \& Communication}: Technical Mentorship, Scientific Writing (5 publications), Public Speaking (27+ conferences), Documentation, DEI Programming
\end{tabitemize}
\vspace{-1.2\baselineskip}
\section*{Research Experience}
\vspace{-0.8\baselineskip}
\begin{longtable}{@{\extracolsep{\fill}}p{0.09\textwidth} p{0.37\textwidth} p{0.30\textwidth} r }
  \textbf{Postdoc} & \textbf{University of South Florida} & \textbf{Prof. David Simmons} & \textbf{2022 -- Present}\\
\end{longtable}
\vspace{-1.2\baselineskip}
\begin{tabitemize}
    \item Simulated thermo-elastoplastic deformation in rubber \& copolymer systems via MD.
    \item Automated rheology pipelines across 500+ simulations \& multi-terabyte datasets, reducing analysis time by 90\%.
    \item Analyzed local stress during dynamic deformation with changing simulation geometry, capturing nanoscale reinforcement mechanisms in polymer composites.
    \item Utilized HPC clusters \& parallel workflows to execute \& process long-timescale simulations ($>$50TB), awarded NSF ACCESS Compute Grant (2023).
    \item Applied Bayesian optimization to model polymer dynamics \& extract glass transitions.
    \item Mentored 11 researchers in HPC, Git, \& molecular simulation.
    \item Presented findings at 17+ venues, earning awards at GRC (2024) \& USF (2023).
\end{tabitemize}
\vspace{-0.7\baselineskip}
\begin{longtable}{@{\extracolsep{\fill}}p{0.09\textwidth} p{0.37\textwidth} p{0.30\textwidth} r }
  \textbf{Ph.D.} & \textbf{Brigham Young University} & \textbf{Prof. Douglas Tree} & \textbf{2017 -- 2022}\\
\end{longtable}
\vspace{-0.7\baselineskip}
\begin{tabitemize}
    \item Developed 2 custom Monte Carlo simulation frameworks (35K+ lines, C++/CUDA), achieving 100$\times$ speedup in modeling polymer crystallization.
    \item Constructed 3D free energy landscapes \& geometric order parameters to analyze molecular deformation \& orientation during structural transitions.
    \item Automated high-throughput parameter sweeps across multi-dimensional polymer systems, accelerating structure-property analysis.
    \item Visualized \& analyzed large 3D trajectories to extract structural \& kinetic insights.
  \item Mentored 4 undergraduates, co-authoring 2 publications \& 6 abstracts.
  \item Played key role in a successful \$500K NSF CAREER proposal.
\end{tabitemize}
\vspace{-0.7\baselineskip}
\begin{longtable}{@{\extracolsep{\fill}}p{0.09\textwidth} p{0.37\textwidth} p{0.30\textwidth} r }
  \textbf{M.S.} & \textbf{American University of Sharjah} & \textbf{Prof. Ghaleb Husseini} & \textbf{2015 -- 2017}\\
\end{longtable}
\vspace{-0.7\baselineskip}
\begin{tabitemize}
  \item Designed polymer-based nanoparticles \& optimized self-assembling formulations to enhance mechanical stability \& controlled drug release.
  \item Standardized lab protocols to improve reproducibility \& cross-team collaboration.
  \item Presented at 3 conferences, earning Best Talk Award at AUS Biomed.~Eng.~Symposium.
\end{tabitemize}
\vspace{-2.1\baselineskip}
\section*{Leadership \& Community Engagement}
\begin{tabitemize}
  \item \textbf{President, Early Career Researchers in Polymer Physics (2022–):} Led a global 550-member community \& organized the 2023 Virtual Symposium with 150+ attendees.
  \item \textbf{President \& Founder, USF Postdoctoral Scholar Association (2023–):} Served 200+ postdocs via career events, DEI initiatives, \& the NPA-funded ELEVATE Talk Series.
  \item \textbf{President \& Founder, BYU Chem.~Eng.~Grad.~Student Council (2019–2022):} Directed recruitment, outreach, \& well-being programs impacting department policy.
\end{tabitemize}
\begin{document}
\begin{center}
  {\LARGE \textbf{Pierre Kawak, Ph.D.} }\\[1ex]
  +1 (801) 762-7999 $\bullet$ \href{mailto:pskawak@gmail.com}{\tt pskawak@gmail.com} $\bullet$ \href{https://linktr.ee/pkawak}{\tt linktr.ee/pkawak}\\
\end{center}
\begin{tabitemize}
  \item Computational software engineer with 8+ years designing HPC simulation tools, scalable pipelines, \& 50TB+ data workflows for physical modeling.
  \item Production-ready experience building modular C++/Python codebases in Linux environments \& automating large-scale data processing across HPC systems.
  \item Passionate about bridging software \& hardware, \& excited to advance Freeform's mission by enabling ML-driven, autonomous manufacturing through backend data infrastructure \& real-time system integration.
\end{tabitemize}
\vspace{-1.0\baselineskip}
\section*{Technical Skills}
\begin{tabitemize}
  \item \textbf{Programming Languages}: Python, C++, CUDA, Bash, C, MATLAB, R
  \item \textbf{Backend \& DevOps}: Modular Design, CI/CD, Unit Testing, Git, Linux/UNIX
  \item \textbf{Data Pipelines \& Infrastructure}: Slurm, Open MPI, Workflow Automation, 50TB+ Data Processing
  \item \textbf{Modeling \& Simulation}: Monte Carlo, Molecular Dynamics (LAMMPS, GROMACS), Thermo-elastoplasticity
  \item \textbf{Data Science \& Visualization}: NumPy, Pandas, scikit-learn, Matplotlib, Bayesian Optimization, 3D Visualization (OVITO, VMD)
  \item \textbf{Debugging \& Profiling}: GDB, Valgrind, nvprof, Callgrind, Memory Leaks
  \item \textbf{Collaboration \& Mentorship:} Technical Mentorship (15+ mentees), Scientific Writing (5 publications), Research Presentations (27+), Documentation
\end{tabitemize}
\vspace{-1.2\baselineskip}
\section*{Research Experience}
\vspace{-0.8\baselineskip}
\begin{longtable}{@{\extracolsep{\fill}}p{0.09\textwidth} p{0.37\textwidth} p{0.30\textwidth} r }
  \textbf{Postdoc} & \textbf{University of South Florida} & \textbf{Prof. David Simmons} & \textbf{2022 -- Present}\\
\end{longtable}
\vspace{-1.2\baselineskip}
\begin{tabitemize}
  \item Developed C++/Python pipelines for thermo-elastoplastic simulation of deformation.
  \item Automated real-time analysis workflows for 50+ TB datasets on HPC clusters using Bash, Slurm and MPI, reducing data processing time by 90\%.
  \item Integrated local stress analysis in dynamic, deformed geometries to study nanoscale reinforcement.
  \item Applied Bayesian optimization to model \& extract material dynamics.
  \item Mentored 11 researchers on HPC workflows, version control, and modular codebases.
  \item Presented findings at 17+ venues; received awards from GRC (2024) and USF (2023).
\end{tabitemize}
\vspace{-0.7\baselineskip}
\begin{longtable}{@{\extracolsep{\fill}}p{0.09\textwidth} p{0.37\textwidth} p{0.30\textwidth} r }
  \textbf{Ph.D.} & \textbf{Brigham Young University} & \textbf{Prof. Douglas Tree} & \textbf{2017 -- 2022}\\
\end{longtable}
\vspace{-1.2\baselineskip}
\begin{tabitemize}
  \item Architected 2 GPU-accelerated Monte Carlo simulation codebase in C++/CUDA (35K+ lines), achieving 100$\times$ speedup in polymer crystallization workflows.
  \item Profiled \& debugged performance bottlenecks using GDB, valgrind, \& nvprof; resolved memory leaks, segmentation faults, \& runtime inefficiencies.
  \item Built modular pipelines for high-throughput structure-property simulations using Git, unit tests, \& dual analysis/production environments.
    \item Visualized \& analyzed large 3D trajectories to extract structural \& kinetic insights.
  \item Mentored 4 undergraduates, co-authoring 2 publications \& 6 abstracts.
  \item Played key role in a successful \$500K NSF CAREER proposal.
\end{tabitemize}
\vspace{-0.7\baselineskip}
\begin{longtable}{@{\extracolsep{\fill}}p{0.09\textwidth} p{0.37\textwidth} p{0.30\textwidth} r }
  \textbf{M.S.} & \textbf{American University of Sharjah} & \textbf{Prof. Ghaleb Husseini} & \textbf{2015 -- 2017}\\
\end{longtable}
\vspace{-1.2\baselineskip}
\begin{tabitemize}
  \item Designed polymer-based nanoparticles \& optimized self-assembling formulations to enhance mechanical stability \& controlled drug release.
  \item Standardized lab protocols to improve reproducibility \& cross-team collaboration.
  \item Presented at 3 conferences, earning Best Talk Award at AUS Biomed.~Eng.~Symposium.
\end{tabitemize}
\vspace{-2.1\baselineskip}
\section*{Leadership \& Community Engagement}
\begin{tabitemize}
  \item \textbf{President, Early Career Researchers in Polymer Physics (2022–):} Led a global 550-member community \& organized the 2023 Virtual Symposium with 150+ attendees.
  \item \textbf{President \& Founder, USF Postdoctoral Scholar Association (2023–):} Served 200+ postdocs via career events, DEI initiatives, \& the NPA-funded ELEVATE Talk Series.
  \item \textbf{President \& Founder, BYU Chem.~Eng.~Grad.~Student Council (2019–2022):} Directed recruitment, outreach, \& well-being programs impacting department policy.
\end{tabitemize}
\begin{document}
\begin{center}
  {\LARGE \textbf{Pierre Kawak, Ph.D.} }\\[1ex]
  +1 (801) 762-7999 $\bullet$ \href{mailto:pskawak@gmail.com}{\tt pskawak@gmail.com} $\bullet$ \href{https://linktr.ee/pkawak}{\tt linktr.ee/pkawak}\\
\end{center}
\begin{tabitemize}
  \item Computational SW engineer with 8+ years of building HPC simulation tools, C++ codebases (40K+ lines), \& scalable pipelines for physical deformation modeling.
  \item Proficient in production-level C++ in Linux environments, debugging \& profiling, \& architecting modular simulation workflows for copolymer \& nanocomposite systems.
  \item Passionate about advancing Freeform's mission by bridging software \& hardware through automation, control integration, \& high-throughput materials computation.
\end{tabitemize}
\vspace{-1.0\baselineskip}
\section*{Technical Skills}
\begin{tabitemize}
  \item \textbf{Programming Languages:} C++, CUDA, Python, Bash, C, MATLAB, R, Make, CMake
  \item \textbf{Software Development:} Modular Design, Unit Testing, Git, Linux, CI/CD Integration
  \item \textbf{Debugging \& Profiling:} GDB, Valgrind, Callgrind, nvprof, Memory Leaks
  \item \textbf{High-Performance Computing (HPC):} Slurm, Open MPI, Parallelization, Workflow Automation, 50TB+ Data Processing
  \item \textbf{Modeling \& Simulation:} Monte Carlo Methods, Molecular Dynamics (LAMMPS, GROMACS), Deformation Modeling, Thermo-elastoplasticity
  \item \textbf{Data Science \& Visualization:} NumPy, Pandas, Matplotlib, scikit-learn, Bayesian Optimization, OVITO, VMD, 3D Visualization
  \item \textbf{Collaboration \& Mentorship:} Technical Mentorship (15+ mentees), Scientific Writing (5 publications), Research Presentations (27+), Documentation
\end{tabitemize}
\vspace{-1.2\baselineskip}
\section*{Research Experience}
\vspace{-0.8\baselineskip}
\begin{longtable}{@{\extracolsep{\fill}}p{0.09\textwidth} p{0.37\textwidth} p{0.30\textwidth} r }
  \textbf{Postdoc} & \textbf{University of South Florida} & \textbf{Prof. David Simmons} & \textbf{2022 -- Present}\\
\end{longtable}
\vspace{-1.2\baselineskip}
\begin{tabitemize}
  \item Developed C++/Python pipelines for thermo-elastoplastic simulation of deformation.
  \item Automated real-time analysis workflows for 50+ TB datasets on HPC clusters using Bash, Slurm and MPI, reducing data processing time by 90\%.
  \item Integrated local stress analysis in dynamic, deformed geometries to study nanoscale reinforcement.
  \item Applied Bayesian optimization to model \& extract material dynamics.
  \item Mentored 11 researchers on HPC workflows, version control, and modular codebases.
  \item Presented findings at 17+ venues; received awards from GRC (2024) and USF (2023).
\end{tabitemize}
\vspace{-0.7\baselineskip}
\begin{longtable}{@{\extracolsep{\fill}}p{0.09\textwidth} p{0.37\textwidth} p{0.30\textwidth} r }
  \textbf{Ph.D.} & \textbf{Brigham Young University} & \textbf{Prof. Douglas Tree} & \textbf{2017 -- 2022}\\
\end{longtable}
\vspace{-1.2\baselineskip}
\begin{tabitemize}
  \item Architected 2 GPU-accelerated Monte Carlo simulation codebase in C++/CUDA (35K+ lines), achieving 100$\times$ speedup in polymer crystallization workflows.
  \item Profiled \& debugged performance bottlenecks using GDB, valgrind, \& nvprof; resolved memory leaks, segmentation faults, \& runtime inefficiencies.
  \item Built modular pipelines for high-throughput structure-property simulations using Git, unit tests, \& dual analysis/production environments.
    \item Visualized \& analyzed large 3D trajectories to extract structural \& kinetic insights.
  \item Mentored 4 undergraduates, co-authoring 2 publications \& 6 abstracts.
  \item Played key role in a successful \$500K NSF CAREER proposal.
\end{tabitemize}
\vspace{-0.7\baselineskip}
\begin{longtable}{@{\extracolsep{\fill}}p{0.09\textwidth} p{0.37\textwidth} p{0.30\textwidth} r }
  \textbf{M.S.} & \textbf{American University of Sharjah} & \textbf{Prof. Ghaleb Husseini} & \textbf{2015 -- 2017}\\
\end{longtable}
\vspace{-0.7\baselineskip}
\begin{tabitemize}
  \item Designed polymer-based nanoparticles \& optimized self-assembling formulations to enhance mechanical stability \& controlled drug release.
  \item Standardized lab protocols to improve reproducibility \& cross-team collaboration.
  \item Presented at 3 conferences, earning Best Talk Award at AUS Biomed.~Eng.~Symposium.
\end{tabitemize}
\vspace{-2.1\baselineskip}
\section*{Leadership \& Community Engagement}
\begin{tabitemize}
  \item \textbf{President, Early Career Researchers in Polymer Physics (2022–):} Led a global 550-member community \& organized the 2023 Virtual Symposium with 150+ attendees.
  \item \textbf{President \& Founder, USF Postdoctoral Scholar Association (2023–):} Served 200+ postdocs via career events, DEI initiatives, \& the NPA-funded ELEVATE Talk Series.
  \item \textbf{President \& Founder, BYU Chem.~Eng.~Grad.~Student Council (2019–2022):} Directed recruitment, outreach, \& well-being programs impacting department policy.
\end{tabitemize}
\begin{document}
\begin{center}
  {\LARGE \textbf{Pierre Kawak, Ph.D.} }\\[1ex]
  +1 (801) 762-7999 $\bullet$ \href{mailto:pskawak@gmail.com}{\tt pskawak@gmail.com} $\bullet$ \href{https://linktr.ee/pkawak}{\tt linktr.ee/pkawak}\\
\end{center}
\begin{tabitemize}
  \item Senior Computational Scientist with 11+ years of experience in molecular modeling, specializing in large-scale simulations \& high-dimensional analysis.
  \item Led collaborative research with polymer chemists to identify sequence-controlled materials with enhanced properties, paralleling the goals of biomolecular design.
  \item Built GPU-accelerated tools (CUDA), automated 50+ TB workflows, \& mentored 15+.
  \item Eager to apply deep expertise in molecular modeling, statistical mechanics, \& high-performance computing (HPC) to ML-guided drug discovery at Genentech.
\end{tabitemize}
\vspace{-1.4\baselineskip}
\section*{Technical Skills}
\begin{tabitemize}
  \item \textbf{Programming \& Software}: Python, C++, CUDA, Bash, Git, Unit Testing, Benchmarking, GDB, Valgrind, Callgrind, nvprof
  \item \textbf{Machine Learning \& Data}: NumPy, Pandas, Scikit-learn, Bayesian Optimization, PyTorch (familiar), Sequence/Structure Data
  \item \textbf{Molecular Modeling \& Simulation}: Statistical Mechanics, Free Energy Methods, Phase Transitions, Coarse-Grained \& Atomistic Systems, VMD, OVITO
  \item \textbf{HPC}: Slurm, OpenMPI, Workflow Automation, Parallel Computing, 50+ TB Pipelines
  \item \textbf{Visualization \& Communication}: Matplotlib, 3D Simulation Data, Scientific Writing (5 publications), Public Speaking (27+ talks), Technical Mentorship (15 trainees)
\end{tabitemize}
\vspace{-1.2\baselineskip}
\section*{Research Experience}
\vspace{-0.7\baselineskip}
\begin{longtable}{@{\extracolsep{\fill}}p{0.09\textwidth} p{0.37\textwidth} p{0.30\textwidth} r }
  \textbf{Postdoc} & \textbf{University of South Florida} & \textbf{Prof. David Simmons} & \textbf{2022 -- Present}\\
\end{longtable}
\vspace{-0.5\baselineskip}
\begin{tabitemize}
  \item Designed polymer sequences with enhanced thermal stability, guiding synthesis via molecular modeling and structure–property prediction.
  \item Applied Bayesian optimization to fit dynamics data and extract glass transition temperatures from relaxation behavior.
  \item Developed trajectory analysis tools enabling 30$\times$ speed-up in post-processing of glassy dynamics data.
  \item Simulated elastomer nanocomposites to uncover nanoscale toughening mechanisms and filler–polymer interactions.
  \item Scaled HPC workflows across 50+ TB of simulations, automating job orchestration, checkpointing, and storage.
  \item Maintained core group codebase and mentored 11 researchers in parallel simulation, reproducibility, and scientific computing.
  \item Delivered 17+ conference talks \& won sci.~comm.~awards at GRC (2024) \& USF (2023).
\end{tabitemize}
\vspace{-1.8\baselineskip}
\begin{longtable}{@{\extracolsep{\fill}}p{0.09\textwidth} p{0.37\textwidth} p{0.30\textwidth} r }
  \textbf{Ph.D.} & \textbf{Brigham Young University} & \textbf{Prof. Douglas Tree} & \textbf{2017 -- 2022}\\
\end{longtable}
\vspace{-1.4\baselineskip}
\begin{tabitemize}
  \item Developed 2 GPU-accelerated Monte Carlo engines (C++/CUDA) (80k+ lines), enabling 100$\times$ faster crystallization studies.
  \item Applied umbrella sampling, Wang–Landau methods, \& novel enhanced sampling techniques to map 3D free energy landscapes of polymer crystallization.
  \item Designed custom order parameters to track structural transitions.
  \item Analyzed large-scale molecular trajectories to extract conformational and kinetic trends.
  \item Mentored 4 researchers, coauthoring 2 papers \& 6 conference abstracts.
  \item Played key role in a successful \$500K NSF CAREER grant.
  \item Earned APS Distinguished Student \& BYU Presentation Award for science comm.
\end{tabitemize}
\vspace{-1.1\baselineskip}
\begin{longtable}{@{\extracolsep{\fill}}p{0.09\textwidth} p{0.37\textwidth} p{0.30\textwidth} r }
  \textbf{M.S.} & \textbf{American University of Sharjah} & \textbf{Prof. Ghaleb Husseini} & \textbf{2015 -- 2017}\\
\end{longtable}
\vspace{-1.2\baselineskip}
\begin{tabitemize}
  \item Engineered estrone-functionalized, ultrasound-sensitive liposomes for targeted breast cancer therapy with tunable drug release profiles.
  \item Standardized lab protocols across teams, boosting reproducibility \& collaboration.
\end{tabitemize}
\vspace{-1.5\baselineskip}
\section*{Leadership \& Community Engagement}
\begin{tabitemize}
  \item \textbf{President, Early Career Researchers in Polymer Physics (2022–):} Led a global 550-member community \& organized the 2023 Virtual Symposium with 150+ attendees.
  \item \textbf{President \& Founder, USF Postdoctoral Scholar Association (2023–):} Served 200+ postdocs via career events, DEI initiatives, \& the NPA-funded ELEVATE Talk Series.
  \item \textbf{President \& Founder, BYU Chem.~Eng.~Grad.~Student Council (2019–2022):} Directed recruitment, outreach, \& well-being programs impacting department policy.
\end{tabitemize}
\vspace{-0.5\baselineskip}
\begin{document}
\begin{center}
  {\LARGE \textbf{Pierre Kawak, Ph.D.} }\\[1ex]
  +1 (801) 762-7999 $\bullet$ \href{mailto:pskawak@gmail.com}{\tt pskawak@gmail.com} $\bullet$ \href{https://linktr.ee/pkawak}{\tt linktr.ee/pkawak}\\
\end{center}
\begin{tabitemize}
  \item Performance-focused software engineer and simulation scientist with 11+ years of experience in C++/Python development, HPC, and performance analysis.
  \item Specialized in system modeling, algorithmic optimization, and distributed architecture workflows across polymers, computational chemistry, and HPC clusters.
  \item Proven track record leading scalable performance tools, driving speedups >100x, and mentoring global research teams.
\end{tabitemize}
\vspace{-1.7\baselineskip}
\section*{Technical Skills}
\begin{tabitemize}
  \item \textbf{Programming}: Python, C++, CUDA, MATLAB, Bash, R
  \item \textbf{Performance \& Simulation}: Monte Carlo, Molecular Dynamics, Free Energy Calculations, LAMMPS, GROMACS
  \item \textbf{Systems \& Architecture}: Distributed Computing, HPC, Slurm, Open MPI, 50TB+ Data Pipelines
  \item \textbf{Performance Engineering}: Bottleneck analysis, Custom modeling frameworks, Optimization of simulation pipelines
  \item \textbf{Tools \& Visualization}: VMD, OVITO, NumPy, Pandas, Matplotlib
  \item \textbf{Leadership \& Communication}: Scientific Writing, Conference Speaking (27+), Team Mentoring (15+ mentees)  
\end{tabitemize}
\vspace{-1.7\baselineskip}
\section*{Research Experience}
\vspace{-0.8\baselineskip}
\begin{longtable}{@{\extracolsep{\fill}}p{0.09\textwidth} p{0.37\textwidth} p{0.30\textwidth} r }
  \textbf{Postdoc} & \textbf{University of South Florida} & \textbf{Prof. David Simmons} & \textbf{2022 -- Present}\\
\end{longtable}
\vspace{-1.5\baselineskip}
\begin{tabitemize}
  \item Developed advanced performance models for polymer deformation systems; improved analysis efficiency by 90\%.
  \item Conducted multi-TB simulations using LAMMPS \& HPC clusters, streamlining distributed workflows across compute nodes.
  \item Engineered custom C++/Python tools to analyze rheology and thermomechanical performance, enabling automated simulations.
  \item Awarded NSF ACCESS HPC grant for high-throughput analysis infrastructure; led architecture and code optimization.
  \item Mentored 11 researchers in HPC system use, Git workflows, and simulation design for scalable distributed projects.
\end{tabitemize}
\vspace{-1.0\baselineskip}
\begin{longtable}{@{\extracolsep{\fill}}p{0.09\textwidth} p{0.37\textwidth} p{0.30\textwidth} r }
  \textbf{Ph.D.} & \textbf{Brigham Young University} & \textbf{Prof. Douglas Tree} & \textbf{2017 -- 2022}\\
\end{longtable}
\vspace{-1.4\baselineskip}
\begin{tabitemize}
  \item Designed two original simulation engines in C++/CUDA, achieving >100$\times$ speedups for polymer crystallization models  
  \item Modeled 3D energy landscapes with free energy and phase transition analysis, optimizing simulation convergence  
  \item Integrated distributed architecture analysis for system-level behavior under changing thermodynamic constraints  
  \item Mentored 4 undergraduates; led publication and conference efforts for high-impact research outcomes  
  \item Played key role in a successful \$500K NSF CAREER proposal targeting advanced computational modeling systems  
\end{tabitemize}
\vspace{-0.7\baselineskip}
\begin{longtable}{@{\extracolsep{\fill}}p{0.09\textwidth} p{0.37\textwidth} p{0.30\textwidth} r }
  \textbf{M.S.} & \textbf{American University of Sharjah} & \textbf{Prof. Ghaleb Husseini} & \textbf{2015 -- 2017}\\
\end{longtable}
\vspace{-1.2\baselineskip}
\begin{tabitemize}
  \item Developed ultrasound-responsive estrone-functionalized liposomal drug carriers for breast cancer treatment
  \item Conducted parameter optimization to model release kinetics and performance under ultrasound-based activation  
  \item Standardized lab performance data pipelines for reproducibility, resulting in 40\% reduction in error rates  
\end{tabitemize}
\vspace{-2.1\baselineskip}
\section*{Leadership \& Community Engagement}
\begin{tabitemize}
  \item \textbf{President, Early Career Researchers in Polymer Physics (2022–):} Led a global 550-member community \& organized the 2023 Virtual Symposium with 150+ attendees.
  \item \textbf{President \& Founder, USF Postdoctoral Scholar Association (2023–):} Served 200+ postdocs via career events, DEI initiatives, \& the NPA-funded ELEVATE Talk Series.
  \item \textbf{President \& Founder, BYU Chem.~Eng.~Grad.~Student Council (2019–2022):} Directed recruitment, outreach, \& well-being programs impacting department policy.
\end{tabitemize}
\begin{document}
\begin{center}
  {\LARGE \textbf{Pierre Kawak, Ph.D.}}\\
  \faPhone\ +1 (801) 762-7999 \quad \faEnvelope\ \href{mailto:pskawak@gmail.com}{pskawak@gmail.com} \quad \faLink\ \href{https://linktr.ee/pkawak}{linktr.ee/pkawak}
\end{center}
\vspace{-0.6\baselineskip}
\section*{Professional Summary}
Innovative computational materials scientist with 11+ years of experience advancing polymer science and sustainable materials through high-impact simulations, novel data science, and cross-disciplinary research. Expert in leveraging high-performance computing (HPC), C++, and Python frameworks to develop scalable tools and drive materials discovery. Proven leadership in research (5 publications), mentorship (16 mentees), science communication (28 talks), and delivering results aligned with ambitious goals.
\vspace{-0.6\baselineskip}
\section*{Technical Skills}
  \textbf{Machine Learning \& AI}: PyTorch, scikit-learn, ML Model Development \\
  \textbf{Programming Languages}: Python, CUDA, C++, MATLAB, Bash, Julia \\
  \textbf{Data Analysis \& Visualization}: Pandas, NumPy, Matplotlib, VMD, OVITO \\
  \textbf{HPC \& Cloud Computing}: Slurm, MPI, Google Cloud, Workflow Automation \\
  \textbf{Scientific Software Engineering}: GitHub Actions, CI/CD, Test-Driven Development \\
  \textbf{Materials Modeling}: Monte Carlo, Molecular Dynamics (LAMMPS, GROMACS), Quantum Chemistry (Gaussian), Atomistic and Coarse-Grained Forcefields
\vspace{-0.6\baselineskip}
\section*{Research Experience}
\textbf{Postdoctoral Researcher}, University of South Florida \hfill \textit{2022 – Present} \\
\textit{Advisor: Prof. David Simmons}
\begin{tabitemize}[leftmargin=*]
  \item Simulated polymer deformation to inform composite design strategies at the nanoscale.
  \item Enhanced copolymer glass transition temperature via sequence-specific simulations; improving thermal stability without altering feedstock or processing.
  \item Developed Python-based rheology tools, improving throughput by 90\%.
  \item Streamlined HPC pipelines processing 50+ TB of data, earning an NSF ACCESS grant.
  \item Mentored 11 researchers; awarded APS Mentoring Fellowship.
  \item Delivered 17 technical talks; received research innovation awards.
\end{tabitemize}
\textbf{Ph.D. Researcher}, Brigham Young University \hfill \textit{2017 – 2022} \\
\textit{Advisor: Prof. Douglas Tree}
\begin{tabitemize}[leftmargin=*]
  \item Developed GPU-accelerated Monte Carlo simulations (C++/CUDA), expediting crystallization studies by 100$\times$.
  \item Created 3D phase diagrams using statistical mechanics and custom order parameters.
  \item Secured APS Distinguished Student Award and BYU Research Presentation Award.
  \item Contributed to successful \$500K NSF CAREER grant.
\end{tabitemize}
\textbf{Graduate Researcher}, American University of Sharjah \hfill \textit{2015 – 2017} \\
\textit{Advisor: Prof. Ghaleb Husseini}
\begin{tabitemize}[leftmargin=*]
  \item Engineered drug delivery nanocarriers; validated dynamic stability via DLS/NMR.
  \item Standardized lab protocols, enhancing reproducibility and cross-lab collaboration.
\end{tabitemize}
\vspace{-0.6\baselineskip}
\section*{Leadership \& Community Engagement}
\begin{tabitemize}[leftmargin=*]
  \item \textbf{President}, Early Career Researchers in Polymer Physics (2022–Present): Led 550+ member global network, organized 150+ attendee virtual symposium.
  \item \textbf{Founder \& President}, USF Postdoctoral Scholar Association (2023–Present): Launched NPA-funded ELEVATE Talk Series and DEI programs for 200+ postdocs.
  \item \textbf{Founder \& President}, BYU Chem.~Eng.~Graduate Council (2019–2022): Shaped department policies and spearheaded outreach and recruitment.
\end{tabitemize}
\vspace{-0.8\baselineskip}
\begin{refsection}[talks]
 \nocite{*}
 \setlength\bibitemsep{0pt}
 \printbibliography[resetnumbers=true,type=inproceedings,title={Selected Presentations},heading=fix]
\end{refsection}
\vspace{-1.0\baselineskip}
\section*{Education}
\textbf{Ph.D.} in Chemical Engineering, Brigham Young University \hfill \textit{2022} \\
\textbf{M.S.} in Chemical Engineering, American University of Sharjah \hfill \textit{2017} \\
\textbf{B.S.} in Chemical Engineering (Econ. Minor), American University of Sharjah \hfill \textit{2015}
\vspace{0.5em}
\noindent\textit{Professional references and full list of publications and presentations available at \href{https://linktr.ee/pkawak}{linktr.ee/pkawak}}
\begin{document}
\begin{center}
  {\LARGE \textbf{Pierre Kawak, Ph.D.} }\\[1ex]
  +1 (801) 762-7999 $\bullet$ \href{mailto:pskawak@gmail.com}{\tt pskawak@gmail.com} $\bullet$ \href{https://linktr.ee/pkawak}{\tt linktr.ee/pkawak}\\
\end{center}
\begin{tabitemize}
  \item Computational software engineer with 12+ years of developing scalable tools, writing 40K+ line C++/CUDA codes, \& optimizing HPC workflows for scientific computing.
  \item Expert in Linux OS dev (C, bash, Python) \& code debugging/profiling (GDB, nvprof), with deep understanding of GPU architecture \& high-throughput data processing.
  \item Strong record of mentoring 15+ junior researchers \& winning national fellowships.
  \item Passionate about accelerating innovation at computation, hardware, \& research nexus.
\end{tabitemize}
\vspace{-2.0\baselineskip}
\section*{Technical Skills}
\begin{tabitemize}
  \item \textbf{Languages \& Programs}: C, C++, CUDA, Open MPI, Python, Bash, MATLAB, R, Git
  \item \textbf{Debugging, Profiling \& Tuning}: GDB, valgrind, Callgrind, nvprof, Slurm
  \item \textbf{Simulation \& Modeling}: LAMMPS, GROMACS, Monte Carlo methods, Molecular Dynamics, Coarse-Graining, Atomistic Modeling, Free Energy Calculations
  \item \textbf{Data Analysis \& Visualization}: NumPy, Pandas, Matplotlib, VMD, OVITO
  \item \textbf{Collaboration \& Mentorship}: Scientific Writing (5 publications), Technical Presentations (27+ conferences), Research Mentoring (15+ mentees), Event Coordination
    % IDEAS: Sci. Communication, 
\end{tabitemize}
\vspace{-1.2\baselineskip}
\section*{Research Experience}
\vspace{-0.4\baselineskip}
\begin{longtable}{@{\extracolsep{\fill}}p{0.09\textwidth} p{0.37\textwidth} p{0.30\textwidth} r }
  \textbf{Postdoc} & \textbf{University of South Florida} & \textbf{Prof. David Simmons} & \textbf{2022 -- Present}\\
\end{longtable}
\vspace{-1.2\baselineskip}
\begin{tabitemize}
  \item Developed C++/Python software to simulate polymer deformation \& thermal stability, accelerating nanocomposite \& copolymer simulation \& design.
  \item Debugged \& profiled group codebases to eliminate bottlenecks \& improve runtime.
  \item Streamlined HPC workflows to handle 50+ TB datasets, cutting runtime by 90\% \& earning NSF ACCESS grant.
  \item Applied Bayesian optimization to fit dynamic models \& extract glass transition temperature ($T_g$), improving model accuracy \& generalizability.
  \item Mentored 15+ researchers on simulation, HPC workflows, \& debugging tools.
  \item Presented at 17+ conferences \& won awards for nanoscale elastomer reinforcement.
\end{tabitemize}
\vspace{-0.7\baselineskip}
\begin{longtable}{@{\extracolsep{\fill}}p{0.09\textwidth} p{0.37\textwidth} p{0.30\textwidth} r }
  \textbf{Ph.D.} & \textbf{Brigham Young University} & \textbf{Prof. Douglas Tree} & \textbf{2017 -- 2022}\\
\end{longtable}
\vspace{-1.0\baselineskip}
\begin{tabitemize}
  \item Developed 40K+ line GPU-accelerated Monte Carlo codebase (in C++/CUDA) from scratch, enabling 100$\times$ speedup in crystallization simulations.
  \item Debugged \& profiled code iteratively using GDB, valgrind, \& nvprof; resolved memory leaks, segmentation faults, \& performance bottlenecks during dev \& scaling.
  \item Built reproducible pipelines using GitHub/Bitbucket, implemented unit testing, \& maintained dual codebases for analysis \& production workflows.
  \item Authored user-facing documentation \& training materials to ensure code longevity.
  \item Analyzed $10^6$+ simulation outputs using custom order metrics, VMD, \& OVITO to identify structural transitions \& kinetic pathways.
  \item Mentored 4 undergraduates, co-authoring 2 papers \& 6 abstracts.
  \item Played key role in \$500K NSF CAREER proposal \& won research \& sci.~comm.~awards.
\end{tabitemize}
\vspace{-1.8\baselineskip}
\begin{longtable}{@{\extracolsep{\fill}}p{0.09\textwidth} p{0.37\textwidth} p{0.30\textwidth} r }
  \textbf{M.S.} & \textbf{American University of Sharjah} & \textbf{Prof. Ghaleb Husseini} & \textbf{2015 -- 2017}\\
\end{longtable}
\vspace{-1.0\baselineskip}
\begin{tabitemize}
  \item Designed ultrasound-triggered drug delivery systems for breast cancer chemotherapy; optimized encapsulation \& release via DLS/NMR.
  \item Standardized lab protocols to improve reproducibility \& cross-team collaboration.
  \item Presented at 3 conferences, earning Best Talk Award at AUS Biomed.~Eng.~Symposium.
\end{tabitemize}
\vspace{-2.1\baselineskip}
\section*{Leadership \& Community Engagement}
\begin{tabitemize}
  \item \textbf{President, Early Career Researchers in Polymer Physics (2022–):} Led a global 550-member community \& organized the 2023 Virtual Symposium with 150+ attendees.
  \item \textbf{President \& Founder, USF Postdoctoral Scholar Association (2023–):} Served 200+ postdocs via career events, DEI initiatives, \& the NPA-funded ELEVATE Talk Series.
  \item \textbf{President \& Founder, BYU Chem.~Eng.~Grad.~Student Council (2019–2022):} Directed recruitment, outreach, \& well-being programs impacting department policy.
\end{tabitemize}
\begin{document}
\begin{center}
  {\LARGE \textbf{Pierre Kawak, Ph.D.} }\\[1ex]
  +1 (801) 762-7999 $\bullet$ \href{mailto:pskawak@gmail.com}{\tt pskawak@gmail.com} $\bullet$ \href{https://linktr.ee/pkawak}{\tt linktr.ee/pkawak}\\
\end{center}
\begin{tabitemize}
  \item Computational materials scientist with 11+ years of experience developing simulation \& ML-based models to accelerate materials R\&D.
  \item Applied molecular dynamics, Bayesian optimization, \& acoustic modeling to predict properties \& guide design, including ultrasound-triggered nanoparticle release.
  \item Passion for building scalable physics tools to improve performance \& manufacturing.
  \item Eager to support Liminal by advancing battery inspection via data-driven modeling.
\end{tabitemize}
\vspace{-1.5\baselineskip}
\section*{Technical Skills}
\begin{tabitemize}
  \item \textbf{Programming \& Computing}: Python, C++, CUDA, Bash, MATLAB, R
  \item \textbf{Simulation}: Molecular Dynamics (LAMMPS, GROMACS), Monte Carlo, Free Energy Calculations, Continuum Modeling, Coarse-Graining, Gaussian, AMBER, OPLS
  \item \textbf{Machine Learning \& Analytics}: Scikit-learn, NumPy, Pandas, Matplotlib, Bayesian Optimization, Regression, Signal Processing, High-Dimensional Data Analysis
  \item \textbf{Ultrasound \& Experimental}: Acoustic Signal Modeling, Frequency Optimization, Drug Encapsulation, Liposomal Nanoparticles, DLS, NMR, Release Kinetics
  \item \textbf{High-Performance Computing}: Slurm, Open MPI, Workflow Automation, Parallelization, 50+ TB Data Pipelines
  \item \textbf{Communication \& Leadership}: Scientific Writing (5 publications), Public Speaking (27+ talks), Mentorship (15+ trainees), DEI Advocacy, Event Coordination
\end{tabitemize}
\vspace{-1.6\baselineskip}
\section*{Research Experience}
\vspace{-1.0\baselineskip}
\begin{longtable}{@{\extracolsep{\fill}}p{0.09\textwidth} p{0.37\textwidth} p{0.30\textwidth} r }
  \textbf{Postdoc} & \textbf{University of South Florida} & \textbf{Prof. David Simmons} & \textbf{2022 -- Present}\\
\end{longtable}
\vspace{-1.4\baselineskip}
\begin{tabitemize}
  \item Developed custom molecular dynamics simulations to simulate nanoscale polymer mechanics, supporting high-throughput analysis of material performance.
  \item Applied Bayesian optimization to extract glass transition temperatures from MD simulations, accelerating design of stable copolymers.
  \item Processed 50+ TB of simulation data via parallelized HPC workflows, reducing turnaround time by 90\% and earning NSF ACCESS grant.
  \item Created analysis tools for stress and rheology behavior, enabling reproducible, physics-informed predictions across composite systems.
  \item Mentored 11+ researchers on simulation, HPC, and version control
  \item Presented findings at 17+ conferences; awarded GRC (2024) and USF Symposium (2023) honors for contributions to polymer analytics.
\end{tabitemize}
\vspace{-1.0\baselineskip}
\begin{longtable}{@{\extracolsep{\fill}}p{0.09\textwidth} p{0.37\textwidth} p{0.30\textwidth} r }
  \textbf{Ph.D.} & \textbf{Brigham Young University} & \textbf{Prof. Douglas Tree} & \textbf{2017 -- 2022}\\
\end{longtable}
\vspace{-1.4\baselineskip}
\begin{tabitemize}
  \item Developed Monte Carlo codes (C++/CUDA) to simulate crystallization, achieving 100$\times$ speedup in computing free energy landscapes.
  \item Modeled phase transitions using physical order parameters, quantifying material structure–property relationships at the nanoscale.
  \item Analyzed 3D datasets with VMD/OVITO to extract crystal kinetics and structure.
  \item Mentored 4 undergraduates, co-authoring 2 publications and 6 conference abstracts.
  \item Played key role in successful \$500K NSF CAREER proposal.
  \item Recognized with APS Distinguished Student Award and BYU Presentation Award.
\end{tabitemize}
\vspace{-0.7\baselineskip}
\begin{longtable}{@{\extracolsep{\fill}}p{0.09\textwidth} p{0.37\textwidth} p{0.30\textwidth} r }
  \textbf{M.S.} & \textbf{American University of Sharjah} & \textbf{Prof. Ghaleb Husseini} & \textbf{2015 -- 2017}\\
\end{longtable}
\vspace{-1.2\baselineskip}
\begin{tabitemize}
  \item Modeled ultrasound-triggered drug release from liposomal nanoparticles, optimizing acoustic parameters for targeted, stable delivery.
  \item Characterized encapsulation via DLS \& NMR, validating clinical performance.
  \item Designed \& synthesized estrone-tethered liposomes for targeted breast cancer therapy.
  \item Improved reproducibility by standardizing protocols across cross-functional teams.
  \item Awarded Best Talk at AUS Biomed.~Eng.~Symp.~for acoustic nanodelivery.
\end{tabitemize}
\vspace{-1.2\baselineskip}
\section*{Leadership \& Community Engagement}
\begin{tabitemize}
  \item \textbf{President, Early Career Researchers in Polymer Physics (2022–):} Led a global 550-member community \& organized the 2023 Virtual Symposium with 150+ attendees.
  \item \textbf{President \& Founder, USF Postdoctoral Scholar Association (2023–):} Served 200+ postdocs via career events, DEI initiatives, \& the NPA-funded ELEVATE Talk Series.
  \item \textbf{President \& Founder, BYU Chem.~Eng.~Grad.~Student Council (2019–2022):} Directed recruitment, outreach, \& well-being programs impacting department policy.
\end{tabitemize}
\begin{document}
\begin{center}
  {\LARGE \textbf{Pierre Kawak, Ph.D.} }\\[1ex]
  +1 (801) 762-7999 $\bullet$ \href{mailto:pskawak@gmail.com}{\tt pskawak@gmail.com} $\bullet$ \href{https://linktr.ee/pkawak}{\tt linktr.ee/pkawak}\\
\end{center}
\vspace{-0.1cm}
Senior Computational Scientist with 7 years of experience in molecular modeling \& simulation, \& 4 years in experimental drug delivery \& liposomal formulation. Expertise spans molecular dynamics (LAMMPS, GROMACS, AMBER), free energy calculations, GPU-accelerated simulations, \& Python-based workflow automation, including:
\vspace{-0.3\baselineskip}
\begin{tabitemize}
  \item Executed high-throughput simulations for nanoscale materials using HPC infrastructure, earning an NSF ACCESS Compute Resource Grant (2023).
  \item Developed ultrasound-sensitive liposomal drug delivery systems \& nanoparticle formulations, improving chemotherapy efficiency \& stability.
  \item Mentored 16+ junior researchers, authored 5 peer-reviewed articles, \& presented at 27+ conferences, with national recognition (APS Career Mentor Fellowship, GRC Best Poster Award, NPA IMPACT Fellowship). Eager to apply computational chemistry \& simulation-driven design to accelerate molecular glue discovery at Neomorph.
\end{tabitemize}
\vspace{-1.5\baselineskip}
\section*{Research Experience}
\vspace{-0.5\baselineskip}
\begin{longtable}{@{\extracolsep{\fill}}p{0.09\textwidth} p{0.37\textwidth} p{0.30\textwidth} r }
  \textbf{Postdoc} & \textbf{University of South Florida} & \textbf{Prof. David Simmons} & \textbf{2022 -- Present}\\
\end{longtable}
\vspace{-1.4\baselineskip}
\begin{tabitemize}
  \item Simulated nanoscale mechanical response of polymer composites using high-throughput molecular dynamics (LAMMPS, OPLS), revealing deformation pathways that inform molecular design of toughened materials.
  \item Engineered a rheology analysis framework of stress–strain from simulations, enabling discovery of nanoscale relaxation mechanisms linked to material failure resistance.
  \item Modeled thermal performance of copolymers across atomistic \& coarse-grained scales, identifying novel sequences with enhanced glass transition temperature $\bm{T_g}$ without altering composition or processing routes.
  \item Automated processing of 50TB+ simulation datasets via Python \& bash, reducing analysis time by 90\% \& securing NSF ACCESS Compute Resource Grant (2023).
  \item Elevated group research capacity by mentoring 11 researchers in molecular simulation, HPC, \& version control, earning the APS Career Mentor Fellowship (2023).
  \item Disseminated findings at 17 int'l venues, receiving poster awards for innovations in computational polymer mechanics (USF Postdoc Symposium 2023, GRC 2024).
\end{tabitemize}
\vspace{-1.2\baselineskip}
\begin{longtable}{@{\extracolsep{\fill}}p{0.09\textwidth} p{0.37\textwidth} p{0.30\textwidth} r }
  \textbf{Ph.D.} & \textbf{Brigham Young University} & \textbf{Prof. Douglas Tree} & \textbf{2017 -- 2022}\\
\end{longtable}
\vspace{-1.4\baselineskip}
\begin{tabitemize}
  \item Accelerated Monte Carlo simulations $100\times$ via GPU-optimized CUDA/C++ pipelines, enabling tractable modeling of polymer crystallization \& experimental comparison.
  \item Deployed automated simulation workflows using Python, Bash, \& R to run parametric sweeps across molecular architectures, facilitating large-scale crystallization studies.
  \item Constructed first-ever 3D free energy landscapes \& derived kinetic pathways for polymer order formation, offering new mechanistic insight inaccessible to classical models.
  \item Designed phase diagrams using custom order parameters to map crystalline \& orientational transitions, supporting predictive modeling of material phase behavior.% with multiple metastable \& stable phases.
  \item Visualized complex molecular configurations with VMD \& OVITO, extracting structural signatures from thousands of simulation frames for downstream analysis.
  \item Mentored 4 undergraduates, co-authoring 2 journal papers \& 6 conference abstracts, supporting their successful entry into competitive graduate programs.
  \item Secured research awards, e.g., APS Forum on Intl.~Physics Distinguished Student Award (2022) \& BYU Grad.~Student Society Professional Presentation Award (2021).
  \item Contributed critical preliminary findings that supported the successful NSF CAREER Award (\$500,000) proposal for continued crystallization research.
\end{tabitemize}
\vspace{-0.7\baselineskip}
\begin{longtable}{@{\extracolsep{\fill}}p{0.09\textwidth} p{0.37\textwidth} p{0.30\textwidth} r }
  \textbf{M.S.} & \textbf{American University of Sharjah} & \textbf{Prof. Ghaleb Husseini} & \textbf{2015 -- 2017}\\
\end{longtable}
\vspace{-1.2\baselineskip}
\begin{tabitemize}
  \item Engineered ultrasound-responsive liposomal drug carriers with estrone surface ligands, enhancing breast cancer targeting \& release precision.
  \item Formulated self-assembling chemotherapy nanoparticles using the dry film method, improving drug encapsulation stability \& sustained release profiles.
  \item Quantified ultrasound-triggered drug release across acoustic parameters, identifying optimal frequencies for controlled release \& clinical viability.
  \item Validated nanoparticle integrity \& drug-loading efficiency through NMR \& dynamic light scattering, ensuring structural robustness \& reproducibility.
  \item Standardized lab protocols to improve reproducibility, collaboration, \& data integrity, increasing research efficiency across teams.
  \item Presented findings at 3 conferences \& peer-reviewed publication, earning Best Talk Award at the AUS Biomed. Eng. Symposium (2016) for innovation in drug delivery.
\end{tabitemize}
\vspace{-1.5\baselineskip}
\section*{Leadership \& Community Engagement}
\vspace{-0.5\baselineskip}
\begin{longtable}{@{\extracolsep{\fill}}p{0.20\textwidth} p{0.585\textwidth} r }
  \textbf{President} & \textbf{Early Career Researchers in Polymer Physics} & \textbf{2022 -- Present}\\
\end{longtable}
\vspace{-1.2\baselineskip}
\begin{tabitemize}
  \item Led a 550-member global Slack community, organizing networking, technical, self-development, \& conference prep events, improving belonging of polymer researchers.
  \item Organized the 2023 Virtual Polymer Physics Symposium, a 2-day intl.~event with 150+ attendees, 4 technical sessions, a DEI discussion, \& a diverse career panel.% Prioritized financially constrained \& intl.~speakers, ensuring equitable research exposure.
\end{tabitemize}
\vspace{-0.7\baselineskip}
\begin{longtable}{@{\extracolsep{\fill}}p{0.28\textwidth} p{0.505\textwidth} r }
  \textbf{President and Founder} & \textbf{USF Postdoctoral Scholar Association} & \textbf{2023 -- Present}\\
\end{longtable}
\vspace{-1.2\baselineskip}
\begin{tabitemize}
  \item Served 200+ postdocs through career programming, networking events, \& advocacy, e.g., ELEVATE Talk Series, funded by NPA IMPACT Fellowship (2023, 6\% acc.~rate).
\end{tabitemize}
\vspace{-0.7\baselineskip}
\begin{longtable}{@{\extracolsep{\fill}}p{0.28\textwidth} p{0.538\textwidth} r }
  \textbf{President and Founder} & \textbf{BYU Chem.~Eng.~Graduate Student Council} & \textbf{2019 -- 2022}\\
\end{longtable}
\vspace{-1.2\baselineskip}
\begin{tabitemize}
  \item Organized dept. recruitment, social \& outreach events, social content, \& financial well-being initiatives, e.g., Recruitment Poster Event (2019–2021) \& BBQ Socials (2018–2021).
  \item Administered a financial health survey to assess graduate student well-being, influencing department policy discussions (2021).
\end{tabitemize}
\begin{document}
\begin{center}
  {\LARGE \textbf{Pierre Kawak, Ph.D.}}\\
  \faPhone\ +1 (801) 762-7999 \quad \faEnvelope\ \href{mailto:pskawak@gmail.com}{pskawak@gmail.com} \quad \faLink\ \href{https://linktr.ee/pkawak}{linktr.ee/pkawak}
\end{center}
\vspace{-0.3\baselineskip}
\section*{Professional Summary}
Computational Engineer with 11+ years of experience developing and deploying high-performance numerical algorithms in Python, C++, and CUDA, specializing in accelerating simulations and analysis of material physics.
Proven expertise in GPU acceleration, performance optimization, technical mentorship, and cross-functional collaboration.
Eager to build efficient, scalable mathematical tools to empower the next-generation scientific and AI computing.
\vspace{-0.3\baselineskip}
\section*{Technical Skills}
\begin{tabitemize}[leftmargin=*]
  \item \textbf{Programming Languages:} Python (NumPy, SciPy, CuPy, PyTorch, Numba, Pandas, Matplotlib, scikit-learn), C, C++, MATLAB, Bash, R, Julia
  \item \textbf{High-Performance Computing:} CUDA, MPI, Slurm, GPU Parallelism, GPU memory optimization, HPC Automation
  \item \textbf{Software Development:} Git, GitHub, Debugging, Profiling, Unit testing, Modular code design, CI/CD
  \item \textbf{Communication \& Leadership:} Scientific writing (5 publications), Technical presentations (27+ Talks), Mentorship, Advocacy and Community Building
\end{tabitemize}
\vspace{-0.3\baselineskip}
\section*{Research Experience}
\textbf{Postdoctoral Researcher}, University of South Florida \hfill \textit{2022 – Present} \\
\textit{Advisor: Prof. David Simmons}
\begin{tabitemize}[leftmargin=*]
  \item Developed numerical algorithms to spatially decompose and reduce atomic coordinate data using pairwise distance computations to probe local stress hotspots.
  \item Utilized Python libraries to model polymer dynamics and screen chemical structures for experimental collaborators.
  \item Developed rheology analysis codebase, creating modular, reusable tools to postprocess 50+ TB datasets; accelerating workflows and boosting team efficiency.
  \item Streamlined HPC pipelines on multi-node CPU and GPU clusters; cutting time by 90\% and earned NSF ACCESS grant.
  \item Mentored 11 researchers in simulations, Git, and HPC; named APS Mentoring Fellow.
  \item Delivered 17 conference talks; earned awards at GRC (2024) and USF Symp.~(2023).
\end{tabitemize}
\textbf{Ph.D. Researcher}, Brigham Young University \hfill \textit{2017 – 2022} \\
\textit{Advisor: Prof. Douglas Tree}
\begin{tabitemize}[leftmargin=*]
  \item Built two GPU-accelerated Monte Carlo codes in C++/CUDA; achieving 100$\times$ speedup.
  \item Optimized data structures and pairwise search for complex atomic moves on GPUs.
  \item Profiled and tuned code using Nsight and nvprof to maximize GPU efficiency.
  \item Developed algorithms to quantify order and compute properties from 3D trajectories.
  \item Mentored 4 undergraduates; co-authored 2 papers and 6 conference abstracts.%; supported grad school admissions.
  \item Won APS Distinguished Student Award and BYU Research Presentation Award.
  \item Contributed key data to a successful \$500K NSF CAREER proposal.
\end{tabitemize}
\textbf{Graduate Researcher}, American University of Sharjah \hfill \textit{2015 – 2017} \\
\textit{Advisor: Prof. Ghaleb Husseini}
\begin{tabitemize}[leftmargin=*]
  \item Synthesized estrone-functionalized drug nanocarriers; enhanced release control for chemotherapy applications.
  \item Validated drug stability \& kinetics with DLS/NMR; optimized ultrasonic parameters.
  \item Standardized lab protocols; boosted reproducibility and cross-lab collaboration.
  \item Presented at 3 conferences; awarded Best Talk at AUS Biomedical Symposium.
\end{tabitemize}
\vspace{-0.7\baselineskip}
\section*{Leadership \& Community Engagement}
\begin{tabitemize}[leftmargin=*]
  \item \textbf{President}, Early Career Researchers in Polymer Physics (2022–Present): Led 550+ member global network, organized 150+ attendee virtual symposium.
  \item \textbf{Founder \& President}, USF Postdoctoral Scholar Association (2023–Present): Launched NPA-funded ELEVATE Talk Series and DEI programs for 200+ postdocs.
  \item \textbf{Founder \& President}, BYU Chem.~Eng.~Graduate Council (2019–2022): Shaped department policies and spearheaded outreach and recruitment.
\end{tabitemize}
\vspace{-0.8\baselineskip}
\section*{Education}
\textbf{Ph.D.} in Chemical Engineering, Brigham Young University \hfill \textit{2022} \\
\textbf{M.S.} in Chemical Engineering, American University of Sharjah \hfill \textit{2017} \\
\textbf{B.S.} in Chemical Engineering (Econ. Minor), American University of Sharjah \hfill \textit{2015}
\vspace{0.5em}
\noindent\textit{Full list of publications and presentations available at \href{https://linktr.ee/pkawak}{linktr.ee/pkawak}}
\begin{document}
\begin{center}
  {\LARGE \textbf{Pierre Kawak, Ph.D.}}\\
  \faPhone\ +1 (801) 762-7999 \quad \faEnvelope\ \href{mailto:pskawak@gmail.com}{pskawak@gmail.com} \quad \faLink\ \href{https://linktr.ee/pkawak}{linktr.ee/pkawak}
\end{center}
\vspace{-0.3\baselineskip}
\section*{Professional Summary}
Engineering leader with 11+ years architecting high-performance, scalable computational solutions in Python, C++, and CUDA.
Proven record driving developer productivity through internal tooling, user-centric design, and cross-functional team leadership.
Passionate about bridging research and production through open-source engagement, community building, and product-driven engineering excellence.
\vspace{-0.3\baselineskip}
\section*{Technical Skills}
\begin{tabitemize}[leftmargin=*]
  \item \textbf{Languages \& Ecosystem:} Python (NumPy, CuPy, Numba, PyTorch), C++, CUDA, MPI, Slurm, Git, CI/CD
  \item \textbf{Product \& Team Leadership:} Team Building, Roadmap Planning, Cross-functional Collaboration, Mentorship
  \item \textbf{Software Engineering:} Modular Codebases, Profiling, GPU Optimization
  \item \textbf{Community Engagement:} Scientific writing (5 publications), Technical presentations (27+ Talks), Advocacy and Community Building
\end{tabitemize}
\vspace{-0.3\baselineskip}
\section*{Research Experience}
\textbf{Postdoctoral Researcher}, University of South Florida \hfill \textit{2022 – Present} \\
\textit{Advisor: Prof. David Simmons}
\begin{tabitemize}[leftmargin=*]
    \item Led a team of 11 researchers building Python analysis pipelines for 50+ TB of simulation data, improving efficiency by 90\%.
    \item Architected reusable HPC tooling for collaborations on multiple supercomputers, improving modularity and user experience and earning NSF ACCESS grant.
    \item Worked in two cross-functional teams to deliver results advancing copolymer and rubber technology.
    \item Delivered 17 conference talks at various venues, earning talk and poster awards.
    \item Led a 550+ member virtual research network, led local 200+ postdoc network, and launched DEI programs for over 200 scientists.
\end{tabitemize} \hfill
\textbf{Ph.D. Researcher}, Brigham Young University \hfill \textit{2017 – 2022} \\
\textit{Advisor: Prof. Douglas Tree}
\begin{tabitemize}[leftmargin=*]
    \item Developed two GPU-accelerated Monte Carlo simulation engines in C++/CUDA, achieving 100$\times$ speedups.
    \item Profiled and optimized GPU workflows with NVIDIA Nsight and nvprof, enhancing performance across CUDA pipelines.
    \item Mentored 4+ junior researchers, formalizing onboarding and training processes.
    \item Won APS Distinguished Student Award and BYU Research Presentation Award.
    \item Contributed computational insights to a \$500K NSF CAREER award proposal.
\end{tabitemize} \hfill
\textbf{Graduate Researcher}, American University of Sharjah \hfill \textit{2015 – 2017} \\
\textit{Advisor: Prof. Ghaleb Husseini}
\begin{tabitemize}[leftmargin=*]
  \item Synthesized estrone-functionalized drug nanocarriers; enhanced release control for chemotherapy applications.
  \item Validated drug stability \& kinetics with DLS/NMR; optimized ultrasonic parameters.
  \item Standardized lab protocols; boosted reproducibility and cross-lab collaboration.
  \item Presented at 3 conferences; awarded Best Talk at AUS Biomedical Symposium.
\end{tabitemize}
\section*{Leadership \& Community Engagement}
\begin{tabitemize}[leftmargin=*]
  \item \textbf{President}, Early Career Researchers in Polymer Physics (2022–Present): Led 550+ member global network, organized 150+ attendee virtual symposium.
  \item \textbf{Founder \& President}, USF Postdoctoral Scholar Association (2023–Present): Launched NPA-funded ELEVATE Talk Series and DEI programs for 200+ postdocs.
  \item \textbf{Founder \& President}, BYU Chem.~Eng.~Graduate Council (2019–2022): Shaped department policies and spearheaded outreach and recruitment.
\end{tabitemize}
\textbf{Ph.D.} in Chemical Engineering, Brigham Young University \hfill \textit{2022} \\
\textbf{M.S.} in Chemical Engineering, American University of Sharjah \hfill \textit{2017} \\
\textbf{B.S.} in Chemical Engineering (Econ. Minor), American University of Sharjah \hfill \textit{2015}
\vspace{0.5em}
\noindent\textit{Full list of publications and presentations available at \href{https://linktr.ee/pkawak}{linktr.ee/pkawak}}
\begin{document}
\begin{center}
  {\LARGE \textbf{Pierre Kawak, Ph.D.}}\\
  \faPhone\ +1 (801) 762-7999 \quad \faEnvelope\ \href{mailto:pskawak@gmail.com}{pskawak@gmail.com} \quad \faLink\ \href{https://linktr.ee/pkawak}{linktr.ee/pkawak}
\end{center}
\vspace{-0.3\baselineskip}
\section*{Professional Summary}
Engineering leader with 11+ years architecting high-performance, scalable computational solutions in Python, C++, and CUDA.
Proven record driving developer productivity through internal tooling, user-centric design, and cross-functional team leadership.
Passionate about bridging research and production through open-source engagement, community building, and product-driven engineering excellence.
\vspace{-0.3\baselineskip}
\section*{Technical Skills}
\begin{tabitemize}[leftmargin=*]
  \item \textbf{Languages \& Ecosystem:} Python (NumPy, CuPy, Numba, PyTorch), C++, CUDA, MPI, Slurm, Git, CI/CD
  \item \textbf{Product \& Team Leadership:} Team Building, Roadmap Planning, Cross-functional Collaboration, Mentorship
  \item \textbf{Software Engineering:} Modular Codebases, Profiling, GPU Optimization
  \item \textbf{Community Engagement:} Scientific writing (5 publications), Technical presentations (27+ Talks), Advocacy and Community Building
\end{tabitemize}
\vspace{-0.3\baselineskip}
\section*{Research Experience}
\textbf{Postdoctoral Researcher}, University of South Florida \hfill \textit{2022 – Present} \\
\textit{Advisor: Prof. David Simmons}
\begin{tabitemize}[leftmargin=*]
    \item Led a team of 11 researchers building Python analysis pipelines for 50+ TB of simulation data, improving efficiency by 90\%.
    \item Architected reusable HPC tooling for collaborations on multiple supercomputers, improving modularity and user experience and earning NSF ACCESS grant.
    \item Worked in two cross-functional teams to deliver results advancing copolymer and rubber technology.
    \item Delivered 17 conference talks at various venues, earning talk and poster awards.
    \item Led a 550+ member virtual research network, led local 200+ postdoc network, and launched DEI programs for over 200 scientists.
\end{tabitemize} \hfill
\textbf{Ph.D. Researcher}, Brigham Young University \hfill \textit{2017 – 2022} \\
\textit{Advisor: Prof. Douglas Tree}
\begin{tabitemize}[leftmargin=*]
    \item Developed two GPU-accelerated Monte Carlo simulation engines in C++/CUDA, achieving 100$\times$ speedups.
    \item Profiled and optimized GPU workflows with NVIDIA Nsight and nvprof, enhancing performance across CUDA pipelines.
    \item Mentored 4+ junior researchers, formalizing onboarding and training processes.
    \item Won APS Distinguished Student Award and BYU Research Presentation Award.
    \item Contributed computational insights to a \$500K NSF CAREER award proposal.
\end{tabitemize} \hfill
\textbf{Graduate Researcher}, American University of Sharjah \hfill \textit{2015 – 2017} \\
\textit{Advisor: Prof. Ghaleb Husseini}
\begin{tabitemize}[leftmargin=*]
  \item Synthesized estrone-functionalized drug nanocarriers; enhanced release control for chemotherapy applications.
  \item Validated drug stability \& kinetics with DLS/NMR; optimized ultrasonic parameters.
  \item Standardized lab protocols; boosted reproducibility and cross-lab collaboration.
  \item Presented at 3 conferences; awarded Best Talk at AUS Biomedical Symposium.
\end{tabitemize}
\section*{Leadership \& Community Engagement}
\begin{tabitemize}[leftmargin=*]
  \item \textbf{President}, Early Career Researchers in Polymer Physics (2022–Present): Led 550+ member global network, organized 150+ attendee virtual symposium.
  \item \textbf{Founder \& President}, USF Postdoctoral Scholar Association (2023–Present): Launched NPA-funded ELEVATE Talk Series and DEI programs for 200+ postdocs.
  \item \textbf{Founder \& President}, BYU Chem.~Eng.~Graduate Council (2019–2022): Shaped department policies and spearheaded outreach and recruitment.
\end{tabitemize}
\textbf{Ph.D.} in Chemical Engineering, Brigham Young University \hfill \textit{2022} \\
\textbf{M.S.} in Chemical Engineering, American University of Sharjah \hfill \textit{2017} \\
\textbf{B.S.} in Chemical Engineering (Econ. Minor), American University of Sharjah \hfill \textit{2015}
\vspace{0.5em}
\noindent\textit{Full list of publications and presentations available at \href{https://linktr.ee/pkawak}{linktr.ee/pkawak}}
\begin{document}
\begin{center}
  {\LARGE \textbf{Pierre Kawak, Ph.D.} }\\[1ex]
  $\bullet$ (801) 762-7999 $\bullet$ pskawak@gmail.com $\bullet$ linktr.ee/pkawak $\bullet$ \\
\end{center}
\vspace{-0.5cm}
\begin{tabitemize}
  \item 7+ years of computational materials expertise in rubber mechanics, polymer reinforcement, free energy analysis, high-performance molecular modeling, nonlinear rheology, and material degradation modeling.
  \item Authored 5 peer-reviewed articles on filled rubber mechanics, polymer dynamics, and material performance optimization.
  \item Developed \& optimized large-scale molecular simulations (LAMMPS, GROMACS, AMBER) to analyze stress-strain behavior, failure, \& reinforcement mechanisms.
  \item Awarded NSF Discover ACCESS Compute Grant (2023) for large-scale simulation studies on rubber durability \& reinforcement.
  \item Presented at 27+ conferences (APS, ACS, AIChE, GRC, etc.) to industry, government, and academic audiences.
  \item Proficient in Python, C++, CUDA, bash, HPC, Slurm, Open MPI, MATLAB, R, and advanced molecular visualization tools (VMD, OVITO).
\end{tabitemize}
\vspace{-1.5\baselineskip}
\section*{Research Experience}
\vspace{-0.7\baselineskip}
\begin{longtable}{@{\extracolsep{\fill}}p{0.34\textwidth} p{0.445\textwidth} r }
  \textbf{Postdoctoral Researcher} & \textbf{University of South Florida (USF)} & \textbf{2022 -- Present}\\
\end{longtable}
\vspace{-0.7\baselineskip}
\begin{tabitemize}
  \item Developed \& implemented large-scale molecular dynamics (MD) simulations in LAMMPS, GROMACS, AMBER, and OPLS to analyze rubber deformation \& relaxation.
  \item Created novel nonlinear rheology analysis techniques, identifying nanoscale reinforcement mechanisms that improve rubber toughness and energy dissipation.
  \item Optimized copolymer thermal stability by simulating coarse-grained \& atomistic sequences, identifying novel sequences with enhanced glass transition temperatures $\bm{T_{g}}$ without changing feedstock or processing conditions.
  \item Leveraged HPC \& parallel computing to perform multi-terabyte MD simulations, securing an NSF Discover ACCESS Compute Resource Grant (2023).
  \item Developed Python, C++, bash, Slurm, Open MPI, \& R automation tools for molecular modeling of polymer dynamics \& mechanics, accelerating team-wide computational workflows, streamlining multi-terabyte data analysis, \& improving research efficiency.
  \item Mentored \& trained 11 researchers in HPC, version control, \& molecular simulations, boosting collaboration, productivity, \& technical skill development \& earning the APS Career Mentor Fellowship (2023).
  \item Presented findings at 17+ institutional, industrial, \& academic conferences, highlighting advancements in rubber \& copolymer technology, as well as polymer theory, \& earning the Outstanding Poster Award at the Gordon Research Conference (2024) \& the USF Annual Postdoctoral Research Symposium Best Poster Award (2023).
\end{tabitemize}
\vspace{-1.0\baselineskip}
\begin{longtable}{@{\extracolsep{\fill}}p{0.34\textwidth} p{0.478\textwidth} r }
  \textbf{Doctoral Researcher} & \textbf{Brigham Young University} & \textbf{2017 -- 2022}\\
\end{longtable}
\vspace{-1.2\baselineskip}
\begin{tabitemize}
  \item Developed, optimized, \& deployed GPU-accelerated Monte Carlo simulations in C/CUDA, achieving $100\times$ speedup in property computes, enabling experimental comparison.
  \item Automated high-throughput simulations using Python, C++, bash, MATLAB, \& R, sweeping multi-dimensional parameters \& accelerating studies of polymer crystals.
  \item Constructed the first-ever 3D free energy landscapes for polymer crystallization, differentiating order-formation pathways inaccessible to classical simulations.% \& resolving a long-standing theoretical controversy on polymer crystallization pathways.
  \item Developed advanced phase diagrams \& applied order parameters for crystalline \& orientational order, quantifying phase transitions in complex molecular landscapes.
  \item Visualized \& analyzed large datasets of 3D molecular configurations using VMD \& OVITO, extracting key structural \& kinetic insights.
  \item Wrote 2 journal articles with 2 mentored undergraduates, supporting their careers.
  \item Secured research awards, e.g., APS Forum on Intl.~Physics Distinguished Student Award (2022) \& BYU Grad.~Student Society Professional Presentation Award (2021).
  \item Presented at 6 hybrid conferences, communicating findings through pandemic.
  \item Directly contributed to an NSF CAREER Award (\$500,000) for continued crystallization research by producing critical preliminary findings.
\end{tabitemize}
\vspace{-1.0\baselineskip}
\begin{longtable}{@{\extracolsep{\fill}}p{0.34\textwidth} p{0.48\textwidth} r }
  \textbf{Masters Researcher} & \textbf{American University of Sharjah} & \textbf{2015 -- 2017}\\
\end{longtable}
\vspace{-1.5\baselineskip}
\begin{tabitemize}
  \item Developed \& characterized polymer-based nanoparticles, applying NMR and DLS assays to validate mechanical stability and crosslinking efficiency.
  \item Optimized self-assembling polymer formulations for controlled molecular interactions and phase behavior.
  \item Standardized lab protocols to improve reproducibility, collaboration, \& data integrity, increasing research efficiency across teams.
  \item Published findings in a peer-reviewed journal \& presented at 3 conferences, winning the AUS Biomedical Engineering Symposium Best Talk Award (2016).
\end{tabitemize}
\vspace{-1.5\baselineskip}
\section*{Leadership \& Community Engagement}
\vspace{-0.7\baselineskip}
\begin{longtable}{@{\extracolsep{\fill}}p{0.20\textwidth} p{0.585\textwidth} r }
  \textbf{President} & \textbf{Early Career Researchers in Polymer Physics} & \textbf{2022 -- Present}\\
\end{longtable}
\vspace{-0.5\baselineskip}
\begin{tabitemize}
  \item Led a 550-member global Slack community, organizing networking, technical, self-development, \& conference prep events, improving belonging of polymer researchers.
  \item Organized the 2023 Virtual Polymer Physics Symposium, a 2-day intl.~event with 150+ attendees, 4 technical sessions, a DEI discussion, \& a diverse career panel.% Prioritized financially constrained \& intl.~speakers, ensuring equitable research exposure.
\end{tabitemize}
\vspace{-1.0\baselineskip}
\begin{longtable}{@{\extracolsep{\fill}}p{0.28\textwidth} p{0.505\textwidth} r }
  \textbf{President and Founder} & \textbf{USF Postdoctoral Scholar Association} & \textbf{2023 -- Present}\\
\end{longtable}
\vspace{-1.5\baselineskip}
\begin{tabitemize}
  \item Served 200+ postdocs through career programming, networking events, \& advocacy, e.g., ELEVATE Talk Series, funded by NPA IMPACT Fellowship (2023, 6\% acc.~rate).
\end{tabitemize}
\vspace{-1.0\baselineskip}
\begin{longtable}{@{\extracolsep{\fill}}p{0.28\textwidth} p{0.538\textwidth} r }
  \textbf{President and Founder} & \textbf{BYU Chem.~Eng.~Graduate Student Council} & \textbf{2019 -- 2022}\\
\end{longtable}
\vspace{-1.5\baselineskip}
\begin{tabitemize}
  \item Organized dept. recruitment, social \& outreach events, social content, \& financial well-being initiatives, e.g., Recruitment Poster Event (2019–2021) \& BBQ Socials (2018–2021).
  \item Administered a financial health survey to assess graduate student well-being, influencing department policy discussions (2021).
\end{tabitemize}
\begin{document}
\begin{center}
  {\LARGE \textbf{Pierre Kawak, Ph.D.} }\\[1ex]
  +1 (801) 762-7999 $\bullet$ \href{mailto:pskawak@gmail.com}{\tt pskawak@gmail.com} $\bullet$ \href{https://linktr.ee/pkawak}{\tt linktr.ee/pkawak}\\
\end{center}
\begin{tabitemize}
  \item Applications Scientist with 11 years of experience in computational modeling, simulation, and scientific communication across drug delivery and materials science.
  \item Skilled in molecular dynamics, free energy methods, and Python-based automation for accelerating discovery in therapeutic systems and polymers.
  \item Collaborated across scientific departments \& organization to design and deliver inclusive scientific events, fostering community and interdisciplinary exchange.
  \item Passionate about supporting collaborative structure-based drug design through cross-functional training, tool development, and workflow optimization.
  \item Recognized for leadership in public-facing scientific communication, with 27+ conference talks and active roles in mentoring, community engagement, and DEI advocacy.
\end{tabitemize}
\vspace{-1.5\baselineskip}
\section*{Technical Skills}
\begin{tabitemize}
  \item \textbf{Programming \& Workflow Automation}: Python, Bash, C++, CUDA, MATLAB, R; Slurm, Open MPI, HPC Workflow Automation, Large-Scale Data Processing (50TB+)
  \item \textbf{Molecular Simulation \& Modeling}: Molecular Dynamics, Free Energy Calculations, Enhanced Sampling Monte Carlo, Coarse-Grained \& Atomistic Models; LAMMPS, GROMACS, Gaussian, AMBER, OPLS
  \item \textbf{Data Analysis \& Visualization}: NumPy, Pandas, Matplotlib, VMD, OVITO; Custom Structure Order Parameters; 3D Free Energy Landscapes
  \item \textbf{Drug Delivery}: Liposomal Formulations, Drug Encapsulation, Ligand Functionalization, DLS, NMR; Nanoparticle Synthesis \& Release Kinetics
  \item \textbf{Communication \& Scientific Leadership}: 27+ Conference Talks, 5 Publications, Mentorship, DEI Advocacy, Community Organizing
\end{tabitemize}
\section*{Research Experience}
\vspace{-0.4\baselineskip}
\begin{longtable}{@{\extracolsep{\fill}}p{0.09\textwidth} p{0.37\textwidth} p{0.30\textwidth} r }
  \textbf{Postdoc} & \textbf{University of South Florida} & \textbf{Prof. David Simmons} & \textbf{2022 -- Present}\\
\end{longtable}
\vspace{-1.2\baselineskip}
\begin{tabitemize}
  \item Automated large-scale molecular dynamics workflows (Python, Slurm) to analyze 50+ TB datasets, accelerating study throughput by 90\%+.
  \item Simulated polymeric systems to enhance thermal stability via sequence-level design.
  \item Built custom rheology and dynamics analysis tools in Python and extended group simulation codebase, improving reproducibility and researcher productivity.
  \item Delivered scientific support \& training in HPC, Git, and simulation tools for 11 mentees, earning the APS Career Mentor Fellowship.
  \item Presented research on structure-property relationships at 17 conferences, receiving awards at GRC (2024) and USF Symposium (2023).
  \item Modeled nanoscale deformation to guide composite material design strategies.
\end{tabitemize}
\vspace{-0.7\baselineskip}
\begin{longtable}{@{\extracolsep{\fill}}p{0.09\textwidth} p{0.37\textwidth} p{0.30\textwidth} r }
  \textbf{Ph.D.} & \textbf{Brigham Young University} & \textbf{Prof. Douglas Tree} & \textbf{2017 -- 2022}\\
\end{longtable}
\vspace{-1.0\baselineskip}
\begin{tabitemize}
  \item Developed 2 Monte Carlo simulation tools (C++/CUDA) to explore polymer crystallization, sampling 3D free energy surfaces 100$\times$ faster.
  \item Designed and executed high-throughput parameter sweeps across multi-dimensional spaces using Python and Bash-based HPC workflows.
  \item Created structure-based order order parameters to quantify structural transitions and construct advanced phase diagrams.
  \item Analyzed large molecular datasets using VMD and OVITO to extract kinetic and spatial features of crystallization pathways.
  \item Mentored 4 undergraduates, co-authoring 2 publications and 6 conference abstracts.
  \item Recognized with APS Distinguished Student Award and BYU Presentation Award for scientific communication and research impact.
  \item Played key role in a successful \$500K NSF CAREER proposal.
\end{tabitemize}
\vspace{-0.7\baselineskip}
\begin{longtable}{@{\extracolsep{\fill}}p{0.09\textwidth} p{0.37\textwidth} p{0.30\textwidth} r }
  \textbf{M.S.} & \textbf{American University of Sharjah} & \textbf{Prof. Ghaleb Husseini} & \textbf{2015 -- 2017}\\
\end{longtable}
\vspace{-1.0\baselineskip}
\begin{tabitemize}
  \item Designed estrone-functionalized liposomes for targeted breast cancer therapy, applying ligand-based delivery concepts to improve drug specificity \& efficacy.
  \item Characterized drug encapsulation \& release kinetics using DLS \& NMR, optimizing ultrasound-triggered release parameters for clinical stability.
  \item Standardized lab protocols to enhance reproducibility \& cross-team collaboration.
  \item Presented at 3 conferences, earning Best Talk Award at AUS Biomed.~Eng.~Symposium.
\end{tabitemize}
\vspace{-1.0\baselineskip}
\section*{Leadership \& Community Engagement}
\begin{tabitemize}
  \item \textbf{President, Early Career Researchers in Polymer Physics (2022–):} Led a global 550-member community \& organized the 2023 Virtual Symposium with 150+ attendees.
  \item \textbf{President \& Founder, USF Postdoctoral Scholar Association (2023–):} Served 200+ postdocs via career events, DEI initiatives, \& the NPA-funded ELEVATE Talk Series.
  \item \textbf{President \& Founder, BYU Chem.~Eng.~Grad.~Student Council (2019–2022):} Directed recruitment, outreach, \& well-being programs impacting department policy.
\end{tabitemize}
\vspace{1.0\baselineskip}
\begin{document}
\begin{center}
  {\LARGE \textbf{Pierre Kawak, Ph.D.} }\\[1ex]
  +1 (801) 762-7999 $\bullet$ \href{mailto:pskawak@gmail.com}{\tt pskawak@gmail.com} $\bullet$ \href{https://linktr.ee/pkawak}{\tt linktr.ee/pkawak}\\
\end{center}
\begin{tabitemize}
  \item Scientific programmer with 11+ years of experience building modular, well-documented tools in Python \& C++ for high-throughput simulation \& analysis in materials science.
  \item Developed 2 object-oriented C++ Monte Carlo codes from scratch \& created collaborative Python frameworks for job submission \& data processing across diverse teams.
  \item Passionate about writing clean, testable code \& translating scientific insight into maintainable software infrastructure.
  \item Seeking to contribute strong software engineering practices \& a deep scientific background to impactful drug discovery.
\end{tabitemize}
\section*{Technical Skills}
\begin{tabitemize}
  \item \textbf{Programming \& Software Development}: Python, C++, C, CUDA, MATLAB, Bash, R, Object-Oriented Programs, Unit Testing, Modular Code Design, Git, Version Control, Scientific Software Design, Codebase Maintenance
  \item \textbf{Scientific Computing \& Simulation}: Monte Carlo, Molecular Dynamics, Free Energy Calculations, LAMMPS, GROMACS, Gaussian, OPLS, Atomistic, Coarse-Graining
  \item \textbf{Analysis \& Visualization}: VMD, OVITO, NumPy, Pandas, Matplotlib, Advanced Scientific Visualization (3D schematics, configuration rendering, scientific illustration)
  \item \textbf{HPC \& Workflow Optimization}: Slurm, Open MPI, Cluster Management, Workflow Automation, Parallelization, Large-Scale Data Processing (50TB+)
  \item \textbf{Scientific Domains}: Material Simulation, Polymer Physics, Crystallization, Nanoparticles, Drug Delivery, Rheology
  \item \textbf{Collaboration \& Communication}: Public Speaking (27+ conferences), Scientific Writing (5 publications), Technical Documentation, Mentoring, DEI Advocacy
\end{tabitemize}
\section*{Research Experience}
\vspace{-0.4\baselineskip}
\begin{longtable}{@{\extracolsep{\fill}}p{0.09\textwidth} p{0.37\textwidth} p{0.30\textwidth} r }
  \textbf{Postdoc} & \textbf{University of South Florida} & \textbf{Prof. David Simmons} & \textbf{2022 -- Present}\\
\end{longtable}
\vspace{-1.2\baselineskip}
\begin{tabitemize}
  \item Lead targeted simulations of nanocomposites \& copolymers, sweeping high-dimensional design spaces (e.g., nanoparticle size, chemistry) to identify optimal performance.
  \item Develop modular Python/bash/C tools for analysis \& job automation, supporting workflows with 500+ sequential/parallel jobs \& 6-month-long simulations.
  \item Document tools extensively \& create structured tutorials to onboard 11 mentees in technical, scientific, \& communication skills.
  \item Streamline large-scale HPC pipelines (50+ TB), reducing analysis time by 90\%+ \& earning an NSF ACCESS award.
  \item Mentor 11 researchers in HPC, Git, \& simulations, earning APS Mentor Fellowship.
  \item Present at 17 conferences, receiving recognition at GRC (2024) \& USF Symposium (2023) for progress on rubber \& copolymer design.
\end{tabitemize}
\vspace{-0.7\baselineskip}
\begin{longtable}{@{\extracolsep{\fill}}p{0.09\textwidth} p{0.37\textwidth} p{0.30\textwidth} r }
  \textbf{Ph.D.} & \textbf{Brigham Young University} & \textbf{Prof. Douglas Tree} & \textbf{2017 -- 2022}\\
\end{longtable}
\vspace{-1.0\baselineskip}
\begin{tabitemize}
  \item Designed \& implemented 2 modular Monte Carlo codes (35K+ lines each in C++/CUDA) to explore crystallization behavior, accelerating discovery 100$\times$.
  \item Applied object-oriented design principles to build reusable simulation modules supporting diverse chemistries \& sampling strategies.
  \item Integrated unit tests to ensure long-term reliability \& minimize regression.% across continuous code evolution.
  \item Constructed 3D phase diagrams using custom crystalline \& orientational order parameters to reveal molecular transition mechanisms.
  \item Analyzed 3D structural data to extract kinetic \& thermodynamic insights.% across crystallization pathways.
  \item Mentored 4 undergraduates, co-authoring 2 papers \& 6 abstracts, supporting careers.
  \item Played key role in a successful \$500K NSF CAREER proposal \& received national awards for scientific communication.
\end{tabitemize}
\vspace{-0.7\baselineskip}
\begin{longtable}{@{\extracolsep{\fill}}p{0.09\textwidth} p{0.37\textwidth} p{0.30\textwidth} r }
  \textbf{M.S.} & \textbf{American University of Sharjah} & \textbf{Prof. Ghaleb Husseini} & \textbf{2015 -- 2017}\\
\end{longtable}
\vspace{-1.0\baselineskip}
\begin{tabitemize}
  \item Engineered liposomal drug carriers with estrone targeting \& ultrasound-triggered release, enhancing delivery control for breast cancer chemotherapy.
  \item Characterized encapsulation \& release kinetics using DLS \& NMR, optimizing ultrasonic parameters for clinical stability \& efficacy.
  \item Standardized lab protocols to boost reproducibility, collaboration, \& data integrity.
  \item Presented at 3 conferences, earning Best Talk Award at AUS Biomed.~Eng.~Symposium.
\end{tabitemize}
\vspace{-1.0\baselineskip}
\section*{Leadership \& Community Engagement}
\begin{tabitemize}
  \item \textbf{President, Early Career Researchers in Polymer Physics (2022–):} Led a global 550-member community \& organized the 2023 Virtual Symposium with 150+ attendees.
  \item \textbf{President \& Founder, USF Postdoctoral Scholar Association (2023–):} Served 200+ postdocs via career events, DEI initiatives, \& the NPA-funded ELEVATE Talk Series.
  \item \textbf{President \& Founder, BYU Chem.~Eng.~Grad.~Student Council (2019–2022):} Directed recruitment, outreach, \& well-being programs impacting department policy.
\end{tabitemize}
\vspace{1.0\baselineskip}
\begin{document}
\begin{center}
  {\LARGE \textbf{Pierre Kawak, Ph.D.} }\\[1ex]
  +1 (801) 762-7999 $\bullet$ \href{mailto:pskawak@gmail.com}{\tt pskawak@gmail.com} $\bullet$ \href{https://linktr.ee/pkawak}{\tt linktr.ee/pkawak}\\
\end{center}
\begin{tabitemize}
  \item Scientific software developer with 11+ years of experience using Python and C++ to build high-throughput molecular simulation tools and force field-based models.
  \item Specialized in atomistic and coarse-grained polymer simulations using OPLS and custom physics-based molecular dynamics and Monte Carlo frameworks to study material properties with experimental accuracy.
  \item Passionate about writing clean, test-driven code and advancing molecular modeling to accelerate discovery in chemistry, materials, and therapeutics.
\end{tabitemize}
\section*{Technical Skills}
\begin{tabitemize}
  \item \textbf{Simulation \& Molecular Modeling}: Atomistic \& Coarse-Grained Simulation, Force Field Parameterization, Monte Carlo Sampling, Molecular Dynamics, Free Energy Calculations, Model Validation, LAMMPS, GROMACS, Gaussian, OPLS, AMBER
  \item \textbf{Programming \& Software Development}: Python, C++, C, CUDA, MATLAB, Bash, R, Object-Oriented Programs, Unit Testing, Modular Code Design, Git, Version Control, Scientific Software Engineering
  \item \textbf{Data Analysis \& Visualization}: VMD, OVITO, NumPy, Pandas, Matplotlib, 3D Scientific Visualization, Simulation Output Parsing
  \item \textbf{HPC \& Workflow Optimization}: Slurm, Open MPI, HPC Cluster Management, Workflow Automation, Parallelization, Large-Scale Data Processing (50TB+)
  \item \textbf{Scientific Domains}: Polymer Physics, Glass Transition, Crystallization, Nanoparticles, Drug Delivery, Rheology
  \item \textbf{Collaboration \& Communication}: Public Speaking (27+ conferences), Scientific Writing (5 publications), Technical Documentation, Mentoring, DEI Advocacy
\end{tabitemize}
\section*{Research Experience}
\vspace{-0.4\baselineskip}
\begin{longtable}{@{\extracolsep{\fill}}p{0.09\textwidth} p{0.37\textwidth} p{0.30\textwidth} r }
  \textbf{Postdoc} & \textbf{University of South Florida} & \textbf{Prof. David Simmons} & \textbf{2022 -- Present}\\
\end{longtable}
\vspace{-1.2\baselineskip}
\begin{tabitemize}
  \item Simulate atomistic copolymers with OPLS force field with high accuracy to identify sequences with enhanced thermal stability without altering feedstock or processing.
  \item Develop modular Python/bash/C tools for rheology analysis with automated workflows spanning 500+ sequential/parallel jobs \& 6-month-long simulations.
  \item Perform high-throughput parameter sweeps across nanoparticle size, volume fraction, monomer chemistry, etc.~to optimize nanocomposite \& copolymer performance.
  \item Document \& validate internal codebases, \& created structured tutorials to onboard 11 mentees in simulation, HPC workflows, \& technical practices.
  \item Streamline large-scale HPC pipelines (50+ TB), reducing analysis time by 90\%+ \& earning an NSF ACCESS award.
  \item Present at 17 conferences, receiving recognition at GRC (2024) \& USF Symposium (2023) for simulation-driven rubber \& copolymer design.
\end{tabitemize}
\vspace{-0.7\baselineskip}
\begin{longtable}{@{\extracolsep{\fill}}p{0.09\textwidth} p{0.37\textwidth} p{0.30\textwidth} r }
  \textbf{Ph.D.} & \textbf{Brigham Young University} & \textbf{Prof. Douglas Tree} & \textbf{2017 -- 2022}\\
\end{longtable}
\vspace{-1.0\baselineskip}
\begin{tabitemize}
  \item Built 2 modular Monte Carlo codes (35K+ lines each in C++/CUDA) to accelerate crystallization modeling (100$\times$) \& explore complex free energy landscapes.
  \item Designed reusable interaction modules to support diverse chemistries and sampling strategies in custom molecular simulations.
  \item Constructed 3D phase diagrams using custom crystalline \& orientational order parameters to reveal molecular transition mechanisms.
  \item Analyzed 3D structural data to extract kinetic \& thermodynamic insights.% across crystallization pathways.
  \item Mentored 4 undergraduates, co-authoring 2 papers \& 6 abstracts, supporting careers.
  \item Played key role in a successful \$500K NSF CAREER proposal \& received national awards for scientific communication.
\end{tabitemize}
\vspace{-0.7\baselineskip}
\begin{longtable}{@{\extracolsep{\fill}}p{0.09\textwidth} p{0.37\textwidth} p{0.30\textwidth} r }
  \textbf{M.S.} & \textbf{American University of Sharjah} & \textbf{Prof. Ghaleb Husseini} & \textbf{2015 -- 2017}\\
\end{longtable}
\vspace{-1.0\baselineskip}
\begin{tabitemize}
  \item Engineered liposomal drug carriers with estrone targeting \& ultrasound-triggered release, enhancing delivery control for breast cancer chemotherapy.
  \item Characterized encapsulation \& release kinetics using DLS \& NMR, gaining insight into nanoscale interactions and delivery efficiency.
  \item Standardized lab protocols to boost reproducibility, collaboration, \& data integrity.
  \item Presented at 3 conferences, earning Best Talk Award at AUS Biomed.~Eng.~Symposium.
\end{tabitemize}
\vspace{-1.0\baselineskip}
\section*{Leadership \& Community Engagement}
\begin{tabitemize}
  \item \textbf{President, Early Career Researchers in Polymer Physics (2022–):} Led a global 550-member community \& organized the 2023 Virtual Symposium with 150+ attendees.
  \item \textbf{President \& Founder, USF Postdoctoral Scholar Association (2023–):} Served 200+ postdocs via career events, DEI initiatives, \& the NPA-funded ELEVATE Talk Series.
  \item \textbf{President \& Founder, BYU Chem.~Eng.~Grad.~Student Council (2019–2022):} Directed recruitment, outreach, \& well-being programs impacting department policy.
\end{tabitemize}
\vspace{1.0\baselineskip}
\begin{document}
\begin{center}
  {\LARGE \textbf{Pierre Kawak, Ph.D.} }\\[1ex]
  +1 (801) 762-7999 $\bullet$ \href{mailto:pskawak@gmail.com}{\tt pskawak@gmail.com} $\bullet$ \href{https://linktr.ee/pkawak}{\tt linktr.ee/pkawak}\\
\end{center}
\begin{tabitemize}
  \item Computational materials scientist with 11+ years of experience in molecular dynamics, polymer simulations, and high-performance computing workflows.
  \item Skilled in Python, C++, CUDA, \& LAMMPS for building scalable, reproducible tools.
  \item Hands-on experience with GPU-accelerated molecular simulations, workflow automation, and data-driven property prediction.
  \item Proven track record in optimizing computational pipelines, mentoring interdisciplinary teams, and contributing to open science through 5 publications and 27+ presentations.
  \item Eager to advance next-generation battery technologies by bridging molecular modeling with AI and HPC at SES AI.
\end{tabitemize}
\section*{Technical Skills}
\begin{tabitemize}
  \item \textbf{Programming \& Automation}: Python, C++, C, CUDA, MATLAB, Bash, R, Git
  \item \textbf{Simulation \& Modeling}: LAMMPS, GROMACS, Gaussian, AMBER, OPLS, Monte Carlo, Molecular Dynamics, Multi-Scale Modeling, Free Energy Calculations
  \item \textbf{Machine Learning \& Optimization}: Bayesian Optimization, Model Fitting, Data-driven Property Prediction
  \item \textbf{HPC \& Workflow Engineering}: Slurm, OpenMPI, Cluster Resource Management, Parallelization, Workflow Automation, Large-Scale Data Processing (50TB+)
  \item \textbf{Data Analysis \& Visualization}: VMD, OVITO, NumPy, Pandas, Matplotlib
  \item \textbf{Experimental Techniques}: Drug Encapsulation, DLS, NMR, Liposomal Formulations, Nanoparticle Synthesis
  \item \textbf{Communication \& Leadership}: Technical Mentorship, DEI Advocacy, Public Speaking (27+ conferences), Scientific Writing (5 publications), Event Coordination
    % IDEAS: Sci. Communication, 
\end{tabitemize}
\section*{Research Experience}
\vspace{-0.4\baselineskip}
\begin{longtable}{@{\extracolsep{\fill}}p{0.09\textwidth} p{0.37\textwidth} p{0.30\textwidth} r }
  \textbf{Postdoc} & \textbf{University of South Florida} & \textbf{Prof. David Simmons} & \textbf{2022 -- Present}\\
\end{longtable}
\vspace{-1.2\baselineskip}
\begin{tabitemize}
  \item Develop, optimize, \& deploy scalable LAMMPS molecular simulations of rubber deformation to extract nanoscale insights for stress response \& composite optimization.
  \item Collaborate with chemists to identify copolymer sequences with enhanced thermal stability via targeted simulations without feedstock or process changes.
  \item Develop custom, rigorous, well-documented mechanics \& dynamics analysis tools in Python/C++ \& integrate them into in-house simulation workflows.
  \item Streamline molecular simulations \& maximize resource utilization via Slurm across HPC clusters of 50+ TB simulation datasets, earning NSF ACCESS grant.
  \item Mentor 11 researchers on Git, HPC scripting, \& simulation methods, earning APS Career Mentor Fellowship.
  \item Present award-winning research at 17 conferences, including GRC 2024 \& USF 2023, on rubber dynamics \& copolymer design.
\end{tabitemize}
\vspace{-0.7\baselineskip}
\begin{longtable}{@{\extracolsep{\fill}}p{0.09\textwidth} p{0.37\textwidth} p{0.30\textwidth} r }
  \textbf{Ph.D.} & \textbf{Brigham Young University} & \textbf{Prof. Douglas Tree} & \textbf{2017 -- 2022}\\
\end{longtable}
\vspace{-1.0\baselineskip}
\begin{tabitemize}
  \item Designed 2 C++/CUDA Monte Carlo simulation packages (internal GitHub repos).%to streamline molecular simulations.
  \item Resolved computational bottlenecks through iterative profiling, debugging, \& code redesign to enable GPU-accelerated, high-accuracy simulations.
  \item Implemented statistical mechanics pipelines to extract free energies from noisy data with rigorous replication strategies.
  \item Analyzed large 3D simulation datasets with custom crystalline \& orientational order parameters to extract key structural \& kinetic insights.
  \item Mentored 4 undergraduates, co-authoring 2 papers \& 6 abstracts, supporting careers.
  \item Contributed data \& writing to a successful \$500K NSF CAREER proposal.
  \item Earned APS Distinguished Student Award (2022) \& BYU Presentation Award (2021) for research excellence \& scientific communication.
\end{tabitemize}
\vspace{-0.7\baselineskip}
\begin{longtable}{@{\extracolsep{\fill}}p{0.09\textwidth} p{0.37\textwidth} p{0.30\textwidth} r }
  \textbf{M.S.} & \textbf{American University of Sharjah} & \textbf{Prof. Ghaleb Husseini} & \textbf{2015 -- 2017}\\
\end{longtable}
\vspace{-1.0\baselineskip}
\begin{tabitemize}
  \item Validated nanoparticle encapsulation \& release kinetics using DLS \& NMR, optimizing formulation parameters \& ultrasonic parameters for breast cancer treatment.
  \item Standardized lab workflows to improve reproducibility, collaboration, \& data quality.
  \item Presented award-winning research at AUS Biomedical Engineering Symposium.
\end{tabitemize}
\section*{Leadership \& Community Engagement}
\begin{tabitemize}
  \item \textbf{President, Early Career Researchers in Polymer Physics (2022–):} Led a global 550-member community \& organized the 2023 Virtual Symposium with 150+ attendees.
  \item \textbf{President \& Founder, USF Postdoctoral Scholar Association (2023–):} Served 200+ postdocs via career events, DEI initiatives, \& the NPA-funded ELEVATE Talk Series.
  \item \textbf{President \& Founder, BYU Chem.~Eng.~Grad.~Student Council (2019–2022):} Directed recruitment, outreach, \& well-being programs impacting department policy.
\end{tabitemize}
\begin{document}
\begin{center}
  {\LARGE \textbf{Pierre Kawak, Ph.D.} }\\[1ex]
  +1 (801) 762-7999 $\bullet$ \href{mailto:pskawak@gmail.com}{\tt pskawak@gmail.com} $\bullet$ \href{https://linktr.ee/pkawak}{\tt linktr.ee/pkawak}\\
\end{center}
\begin{tabitemize}
  \item Computational SW engineer with 12+ years of experience architecting scalable C++, CUDA simulation tools \& optimizing parallel workflows for engineering applications.
  \item Proven ability to implement solutions for open-ended challenges, from developing 40K+ line GPU-accelerated codes to streamlining HPC pipelines to process 50TB data.
  \item Adept at code optimization, debugging (GDB, valgrind, nvprof), \& performance tuning for complex algorithms across UNIX/Linux environments.
  \item Collaborative leader \& mentor to 15+ researchers, with a strong publication \& presentation record in polymer simulation \& algorithm development.
  \item Passionate about applying technical depth \& creativity to accelerate innovation at the nexus of simulation performance, software engineering, \& scientific discovery.
\end{tabitemize}
\vspace{-2.0\baselineskip}
\section*{Technical Skills}
\begin{tabitemize}
  \item \textbf{Programming \& Scripting:} C, C++, GCC, CUDA, NVCC, Make, CMake, Python, Bash, MATLAB, R
  \item \textbf{Parallelism \& HPC:} Open MPI, Slurm, GPU Acceleration, Workflow Optimization
  \item \textbf{Debugging \& Profiling:} GDB, Valgrind, Callgrind, nvprof, Performance Tuning
  \item \textbf{Simulation \& Modeling:} Monte Carlo, Molecular Dynamics (LAMMPS, GROMACS)
  \item \textbf{Data Analysis \& Visualization:} NumPy, Pandas, Matplotlib, VMD, OVITO
  \item \textbf{Version Control \& Collaboration:} GIT, GitHub, Documentation, Mentorship
\end{tabitemize}
\vspace{-1.2\baselineskip}
\section*{Research Experience}
\vspace{-0.4\baselineskip}
\begin{longtable}{@{\extracolsep{\fill}}p{0.09\textwidth} p{0.37\textwidth} p{0.30\textwidth} r }
  \textbf{Postdoc} & \textbf{University of South Florida} & \textbf{Prof. David Simmons} & \textbf{2022 -- Present}\\
\end{longtable}
\vspace{-1.2\baselineskip}
\begin{tabitemize}
  \item Architected C++/Python simulation tools for polymer deformation and thermal stability, enabling fast iteration across copolymer and rubber design space.
  \item Optimized high-throughput workflows to process 50+ TB datasets on HPC clusters, reducing runtime by 90\% and earning NSF ACCESS grant.
  \item Profiled and debugged large C++ codebases using GDB and valgrind to resolve memory leaks, segmentation faults, and scaling inefficiencies.
  \item Applied Bayesian optimization to model dynamic behavior \& extract glass transitions.
  \item Mentored 15+ researchers on simulation, profiling, and scientific computing, fostering collaboration and increasing group velocity.
  \item Led internal documentation and technical presentations to standardize workflows and sustain long-term code maintainability.
\end{tabitemize}
\vspace{-0.7\baselineskip}
\begin{longtable}{@{\extracolsep{\fill}}p{0.09\textwidth} p{0.37\textwidth} p{0.30\textwidth} r }
  \textbf{Ph.D.} & \textbf{Brigham Young University} & \textbf{Prof. Douglas Tree} & \textbf{2017 -- 2022}\\
\end{longtable}
\vspace{-1.0\baselineskip}
\begin{tabitemize}
  \item Iteratively developed, debugged, \& profiled 2 40K+ line C++/CUDA Monte Carlo engine from scratch, enabling 100$\times$ speedup in crystallization simulations.
  \item Built reproducible workflows with unit testing for analysis \& production codebases.
  \item Authored modular documentation \& user guides to improve long-term maintainability \& ease of onboarding.
  \item Analyzed $10^6$+ trajectories to quantify order parameters \& detect kinetic transitions.
  \item Mentored 4 undergraduates, co-authoring 2 publications \& 6 abstracts.
  \item Played key role in a successful \$500K NSF CAREER proposal.
\end{tabitemize}
\vspace{-0.7\baselineskip}
\begin{longtable}{@{\extracolsep{\fill}}p{0.09\textwidth} p{0.37\textwidth} p{0.30\textwidth} r }
  \textbf{M.S.} & \textbf{American University of Sharjah} & \textbf{Prof. Ghaleb Husseini} & \textbf{2015 -- 2017}\\
\end{longtable}
\vspace{-1.0\baselineskip}
\begin{tabitemize}
  \item Designed ultrasound-triggered drug delivery systems \& optimized nanoparticle stability through parameter tuning \& reproducible workflows.
  \item Standardized lab protocols to improve reproducibility \& cross-team collaboration.
  \item Presented at 3 conferences, earning Best Talk Award at AUS Biomed.~Eng.~Symposium.
\end{tabitemize}
\vspace{-2.1\baselineskip}
\section*{Leadership \& Community Engagement}
\begin{tabitemize}
  \item \textbf{President, Early Career Researchers in Polymer Physics (2022–):} Led a global 550-member community \& organized the 2023 Virtual Symposium with 150+ attendees.
  \item \textbf{President \& Founder, USF Postdoctoral Scholar Association (2023–):} Served 200+ postdocs via career events, DEI initiatives, \& the NPA-funded ELEVATE Talk Series.
  \item \textbf{President \& Founder, BYU Chem.~Eng.~Grad.~Student Council (2019–2022):} Directed recruitment, outreach, \& well-being programs impacting department policy.
\end{tabitemize}
