#
- Simulated polymer glassy dynamics, identifying sequences with enhanced thermal stibility \& guiding synthesis by collaborating with chemists.
- Spearheaded a 30$\times$ speed-up in polymer dynamics analysis via bottleneck resolution.
- Extended group's molecular simulation codebase, improving researcher onboarding.
- Automated large-scale HPC workflows across >50 TB of simulation data.
- Mentored 11 researchers in HPC, simulation, Git, and analysis, strengthening lab capabilities and documentation.
- Delivered 17+ conference talks \& won sci.~comm.~awards at GRC (2024) \& USF (2023).
#
- Led multi-variable modeling \& simulation campaigns on nanocomposites \& copolymers, optimizing materials design via high-dimensional parameter sweeps.
- Architected \& maintained HPC simulation \& analysis pipelines processing 50+ TB of data, reducing analysis time by 90\% \& earning NSF ACCESS compute resource award.
- Mentored \& trained 11 researchers in simulation methods, Git workflows, \& job automation, earning APS Career Mentor Fellowship.
- Created comprehensive technical documentation \& onboarding tutorials.
- Delivered 17 talks at conferences; recognized at USF (2023) \& GRC (2024).
#
- Automated sequence-specific copolymer screening to identify formulations with elevated glass transition temperatures ($T_g$).
- Simulated deformation via molecular dynamics to create design rules for composites.
- Curated \& analyzed >50TB of simulation data using parallel Python/C++ pipelines, improving study turnaround time by 90\% and securing NSF HPC grant.
- Mentored 15+ researchers in simulation, Git workflows, \& cloud computing practices; selected for APS Career Mentorship Fellowship.
- Delivered 17+ conference talks \& won sci.~comm.~awards at GRC (2024) \& USF (2023).
#
- Simulated deformation to enable predictive design of reinforced rubber composites.
- Built C++/Python tools to standardize analysis of rheology and chain dynamics.
- Automated HPC workflows to produce \& process 50+ TB of simulation data, reducing turnaround time by 90\%.
- Led structured mentorship for 11+ researchers, including workshops, code development, and onboarding documentation.
- Collaborated with polymer chemists to design \& evaluate copolymer sequences, guiding synthesis of thermally stable materials without feedstock or process changes.
- Presented research at 17+ venues; awarded at USF (2023) \& GRC (2024) meetings for advancements in rubber \& copolymer technology.
#
- Developed multiscale MD models using LAMMPS \& GROMACS to investigate stress relaxation \& nanoscale reinforcement in polymer composites, enabling predictive links between molecular structure \& mechanical performance.
- Created novel Python-based analysis methods for nonlinear rheology \& deformation behavior, supporting simulation-driven property optimization of filled rubber.
- Simulated copolymer thermal behavior using atomistic \& coarse-grained MD (OPLS), identifying high-$\bm{T_g}$ formulations with statistical validation without changing processing or feedstock chemistry, validating performance outcomes through structure-property modeling \& cross-functional collaboration with experimentalists.
- Optimized 50TB+ HPC workflows in bash/Python, improving simulation throughput by 90\% \& enabling rapid iteration across product development cycles; awarded NSF ACCESS Compute Resource Grant (2023).
- Mentored 11 junior researchers in HPC simulation \& data workflows, boosting collaboration, productivity, \& earning APS Career Mentor Fellowship (2023).
- Presented findings at 17 industrial, \& academic conferences, highlighting advancements in polymer characterization \& modeling \& earning poster awards at the Gordon Research Conference (2024) \& the USF Postdoc Research Symposium (2023).
#
- Developed \& applied first-principles molecular dynamics (MD) simulations using LAMMPS \& GROMACS to probe stress relaxation \& deformation mechanics in polymer composites, yielding physical predictions to guide macroscopic performance.
- Designed custom Python-based analysis frameworks to extract nonlinear rheological behavior from simulation data, linking simulation outputs to phenomenological insights relevant for advanced rubber design \& performance/process optimization.
- Simulated coarse-grained \& atomistic copolymer sequences using OPLS \& multi-scale MD approaches to identify $\bm{T_g}$-enhancing formulations, enabling improved thermal properties without feedstock or process changes.
- Architected HPC workflows using bash \& Python to process >50TB of simulation data, reducing data pipeline runtime by 90\% \& enabling real-time iteration across modeling efforts; awarded NSF Discover ACCESS Compute Grant (2023).
- Mentored \& trained 11 researchers in HPC-enabled simulation, version control (Git), \& data management practices, strengthening team modeling capacity \& earning the APS Career Mentor Fellowship (2023).
- Presented modeling innovations \& materials simulation \& theory advancements at 17+ academic \& industry venues, earning Outstanding Poster honors at Gordon Research Conference (2024) \& USF Postdoctoral Symposium (2023) for contributions to predictive polymer performance modeling.
#
- Developed \& applied first-principles molecular dynamics (MD) simulations using LAMMPS \& GROMACS to probe stress relaxation \& deformation mechanics in polymer composites, yielding physical predictions to guide macroscopic performance.
- Designed custom Python-based analysis frameworks to extract nonlinear rheological behavior from simulation data, linking simulation outputs to phenomenological insights relevant for advanced rubber design \& performance/process optimization.
- Simulated coarse-grained \& atomistic copolymer sequences using OPLS \& multi-scale MD approaches to identify $\bm{T_g}$-enhancing formulations, enabling improved thermal properties without feedstock or process changes.
- Architected HPC workflows using bash \& Python to process >50TB of simulation data, reducing data pipeline runtime by 90\% \& enabling real-time iteration across modeling efforts; awarded NSF Discover ACCESS Compute Grant (2023).
- Mentored \& trained 11 researchers in HPC-enabled simulation, version control (Git), \& data management practices, strengthening team modeling capacity \& earning the APS Career Mentor Fellowship (2023).
- Presented modeling innovations \& materials simulation \& theory advancements at 17+ academic \& industry venues, earning Outstanding Poster honors at Gordon Research Conference (2024) \& USF Postdoctoral Symposium (2023) for contributions to predictive polymer performance modeling.
#
- Developed and implemented large-scale stochastic simulations using Python, C++, and CUDA to model complex system dynamics, securing an NSF Discover ACCESS Compute Resource Grant (2023) and leveraging high-performance computing (HPC) to process multi-terabyte datasets efficiently.
- Designed statistical models and optimization algorithms to analyze nonlinear system behaviors and high-dimensional parameter estimation, accelerating simulations by 100$\times$ through parallel computing.
- Applied machine learning techniques to extract predictive insights from complex datasets, improving model performance and decision-making.
- Built and automated data analysis pipelines using Python, C++, bash, and R, streamlining large-scale computational workflows.
- Mentored \& trained 11 researchers in HPC, version control, \& algorithm development, boosting collaboration, productivity, \& earning APS Career Mentor Fellowship (2023).
- Presented findings at 17 institutional, industrial, \& academic conferences, highlighting advancements in rubber \& copolymer technology, as well as polymer theory, \& earning the Outstanding Poster Award at the Gordon Research Conference (2024) \& the USF Annual Postdoctoral Research Symposium Best Poster Award (2023).
#
- Simulated thermo-elastoplastic deformation in rubber \& copolymer systems via MD.
- Automated rheology pipelines across 500+ simulations \& multi-terabyte datasets, reducing analysis time by 90\%.
- Analyzed local stress during dynamic deformation with changing simulation geometry, capturing nanoscale reinforcement mechanisms in polymer composites.
- Utilized HPC clusters \& parallel workflows to execute \& process long-timescale simulations ($>$50TB), awarded NSF ACCESS Compute Grant (2023).
- Applied Bayesian optimization to model polymer dynamics \& extract glass transitions.
- Mentored 11 researchers in HPC, Git, \& molecular simulation.
- Presented findings at 17+ venues, earning awards at GRC (2024) \& USF (2023).
#
- Developed C++/Python pipelines for thermo-elastoplastic simulation of deformation.
- Automated real-time analysis workflows for 50+ TB datasets on HPC clusters using Bash, Slurm and MPI, reducing data processing time by 90\%.
- Integrated local stress analysis in dynamic, deformed geometries to study nanoscale reinforcement.
- Applied Bayesian optimization to model \& extract material dynamics.
- Mentored 11 researchers on HPC workflows, version control, and modular codebases.
- Presented findings at 17+ venues; received awards from GRC (2024) and USF (2023).
#
- Developed C++/Python pipelines for thermo-elastoplastic simulation of deformation.
- Automated real-time analysis workflows for 50+ TB datasets on HPC clusters using Bash, Slurm and MPI, reducing data processing time by 90\%.
- Integrated local stress analysis in dynamic, deformed geometries to study nanoscale reinforcement.
- Applied Bayesian optimization to model \& extract material dynamics.
- Mentored 11 researchers on HPC workflows, version control, and modular codebases.
- Presented findings at 17+ venues; received awards from GRC (2024) and USF (2023).
#
- Designed polymer sequences with enhanced thermal stability, guiding synthesis via molecular modeling and structure–property prediction.
- Applied Bayesian optimization to fit dynamics data and extract glass transition temperatures from relaxation behavior.
- Developed trajectory analysis tools enabling 30$\times$ speed-up in post-processing of glassy dynamics data.
- Simulated elastomer nanocomposites to uncover nanoscale toughening mechanisms and filler–polymer interactions.
- Scaled HPC workflows across 50+ TB of simulations, automating job orchestration, checkpointing, and storage.
- Maintained core group codebase and mentored 11 researchers in parallel simulation, reproducibility, and scientific computing.
- Delivered 17+ conference talks \& won sci.~comm.~awards at GRC (2024) \& USF (2023).
#
- Developed advanced performance models for polymer deformation systems; improved analysis efficiency by 90\%.
- Conducted multi-TB simulations using LAMMPS \& HPC clusters, streamlining distributed workflows across compute nodes.
- Engineered custom C++/Python tools to analyze rheology and thermomechanical performance, enabling automated simulations.
- Awarded NSF ACCESS HPC grant for high-throughput analysis infrastructure; led architecture and code optimization.
- Mentored 11 researchers in HPC system use, Git workflows, and simulation design for scalable distributed projects.
#[leftmargin=*]
- Simulated polymer deformation to inform composite design strategies at the nanoscale.
- Enhanced copolymer glass transition temperature via sequence-specific simulations; improving thermal stability without altering feedstock or processing.
- Developed Python-based rheology tools, improving throughput by 90\%.
- Streamlined HPC pipelines processing 50+ TB of data, earning an NSF ACCESS grant.
- Mentored 11 researchers; awarded APS Mentoring Fellowship.
- Delivered 17 technical talks; received research innovation awards.
#
- Developed C++/Python software to simulate polymer deformation \& thermal stability, accelerating nanocomposite \& copolymer simulation \& design.
- Debugged \& profiled group codebases to eliminate bottlenecks \& improve runtime.
- Streamlined HPC workflows to handle 50+ TB datasets, cutting runtime by 90\% \& earning NSF ACCESS grant.
- Applied Bayesian optimization to fit dynamic models \& extract glass transition temperature ($T_g$), improving model accuracy \& generalizability.
- Mentored 15+ researchers on simulation, HPC workflows, \& debugging tools.
- Presented at 17+ conferences \& won awards for nanoscale elastomer reinforcement.
#
- Developed custom molecular dynamics simulations to simulate nanoscale polymer mechanics, supporting high-throughput analysis of material performance.
- Applied Bayesian optimization to extract glass transition temperatures from MD simulations, accelerating design of stable copolymers.
- Processed 50+ TB of simulation data via parallelized HPC workflows, reducing turnaround time by 90\% and earning NSF ACCESS grant.
- Created analysis tools for stress and rheology behavior, enabling reproducible, physics-informed predictions across composite systems.
- Mentored 11+ researchers on simulation, HPC, and version control
- Presented findings at 17+ conferences; awarded GRC (2024) and USF Symposium (2023) honors for contributions to polymer analytics.
#
- Simulated nanoscale mechanical response of polymer composites using high-throughput molecular dynamics (LAMMPS, OPLS), revealing deformation pathways that inform molecular design of toughened materials.
- Engineered a rheology analysis framework of stress–strain from simulations, enabling discovery of nanoscale relaxation mechanisms linked to material failure resistance.
- Modeled thermal performance of copolymers across atomistic \& coarse-grained scales, identifying novel sequences with enhanced glass transition temperature $\bm{T_g}$ without altering composition or processing routes.
- Automated processing of 50TB+ simulation datasets via Python \& bash, reducing analysis time by 90\% \& securing NSF ACCESS Compute Resource Grant (2023).
- Elevated group research capacity by mentoring 11 researchers in molecular simulation, HPC, \& version control, earning the APS Career Mentor Fellowship (2023).
- Disseminated findings at 17 int'l venues, receiving poster awards for innovations in computational polymer mechanics (USF Postdoc Symposium 2023, GRC 2024).
#[leftmargin=*]
- Developed numerical algorithms to spatially decompose and reduce atomic coordinate data using pairwise distance computations to probe local stress hotspots.
- Utilized Python libraries to model polymer dynamics and screen chemical structures for experimental collaborators.
- Developed rheology analysis codebase, creating modular, reusable tools to postprocess 50+ TB datasets; accelerating workflows and boosting team efficiency.
- Streamlined HPC pipelines on multi-node CPU and GPU clusters; cutting time by 90\% and earned NSF ACCESS grant.
- Mentored 11 researchers in simulations, Git, and HPC; named APS Mentoring Fellow.
- Delivered 17 conference talks; earned awards at GRC (2024) and USF Symp.~(2023).
#[leftmargin=*]
- Led a team of 11 researchers building Python analysis pipelines for 50+ TB of simulation data, improving efficiency by 90\%.
- Architected reusable HPC tooling for collaborations on multiple supercomputers, improving modularity and user experience and earning NSF ACCESS grant.
- Worked in two cross-functional teams to deliver results advancing copolymer and rubber technology.
- Delivered 17 conference talks at various venues, earning talk and poster awards.
- Led a 550+ member virtual research network, led local 200+ postdoc network, and launched DEI programs for over 200 scientists.
#[leftmargin=*]
- Led a team of 11 researchers building Python analysis pipelines for 50+ TB of simulation data, improving efficiency by 90\%.
- Architected reusable HPC tooling for collaborations on multiple supercomputers, improving modularity and user experience and earning NSF ACCESS grant.
- Worked in two cross-functional teams to deliver results advancing copolymer and rubber technology.
- Delivered 17 conference talks at various venues, earning talk and poster awards.
- Led a 550+ member virtual research network, led local 200+ postdoc network, and launched DEI programs for over 200 scientists.
#
- Developed \& implemented large-scale molecular dynamics (MD) simulations in LAMMPS, GROMACS, AMBER, and OPLS to analyze rubber deformation \& relaxation.
- Created novel nonlinear rheology analysis techniques, identifying nanoscale reinforcement mechanisms that improve rubber toughness and energy dissipation.
- Optimized copolymer thermal stability by simulating coarse-grained \& atomistic sequences, identifying novel sequences with enhanced glass transition temperatures $\bm{T_{g}}$ without changing feedstock or processing conditions.
- Leveraged HPC \& parallel computing to perform multi-terabyte MD simulations, securing an NSF Discover ACCESS Compute Resource Grant (2023).
- Developed Python, C++, bash, Slurm, Open MPI, \& R automation tools for molecular modeling of polymer dynamics \& mechanics, accelerating team-wide computational workflows, streamlining multi-terabyte data analysis, \& improving research efficiency.
- Mentored \& trained 11 researchers in HPC, version control, \& molecular simulations, boosting collaboration, productivity, \& technical skill development \& earning the APS Career Mentor Fellowship (2023).
- Presented findings at 17+ institutional, industrial, \& academic conferences, highlighting advancements in rubber \& copolymer technology, as well as polymer theory, \& earning the Outstanding Poster Award at the Gordon Research Conference (2024) \& the USF Annual Postdoctoral Research Symposium Best Poster Award (2023).
#
- Automated large-scale molecular dynamics workflows (Python, Slurm) to analyze 50+ TB datasets, accelerating study throughput by 90\%+.
- Simulated polymeric systems to enhance thermal stability via sequence-level design.
- Built custom rheology and dynamics analysis tools in Python and extended group simulation codebase, improving reproducibility and researcher productivity.
- Delivered scientific support \& training in HPC, Git, and simulation tools for 11 mentees, earning the APS Career Mentor Fellowship.
- Presented research on structure-property relationships at 17 conferences, receiving awards at GRC (2024) and USF Symposium (2023).
- Modeled nanoscale deformation to guide composite material design strategies.
#
- Lead targeted simulations of nanocomposites \& copolymers, sweeping high-dimensional design spaces (e.g., nanoparticle size, chemistry) to identify optimal performance.
- Develop modular Python/bash/C tools for analysis \& job automation, supporting workflows with 500+ sequential/parallel jobs \& 6-month-long simulations.
- Document tools extensively \& create structured tutorials to onboard 11 mentees in technical, scientific, \& communication skills.
- Streamline large-scale HPC pipelines (50+ TB), reducing analysis time by 90\%+ \& earning an NSF ACCESS award.
- Mentor 11 researchers in HPC, Git, \& simulations, earning APS Mentor Fellowship.
- Present at 17 conferences, receiving recognition at GRC (2024) \& USF Symposium (2023) for progress on rubber \& copolymer design.
#
- Simulate atomistic copolymers with OPLS force field with high accuracy to identify sequences with enhanced thermal stability without altering feedstock or processing.
- Develop modular Python/bash/C tools for rheology analysis with automated workflows spanning 500+ sequential/parallel jobs \& 6-month-long simulations.
- Perform high-throughput parameter sweeps across nanoparticle size, volume fraction, monomer chemistry, etc.~to optimize nanocomposite \& copolymer performance.
- Document \& validate internal codebases, \& created structured tutorials to onboard 11 mentees in simulation, HPC workflows, \& technical practices.
- Streamline large-scale HPC pipelines (50+ TB), reducing analysis time by 90\%+ \& earning an NSF ACCESS award.
- Present at 17 conferences, receiving recognition at GRC (2024) \& USF Symposium (2023) for simulation-driven rubber \& copolymer design.
#
- Develop, optimize, \& deploy scalable LAMMPS molecular simulations of rubber deformation to extract nanoscale insights for stress response \& composite optimization.
- Collaborate with chemists to identify copolymer sequences with enhanced thermal stability via targeted simulations without feedstock or process changes.
- Develop custom, rigorous, well-documented mechanics \& dynamics analysis tools in Python/C++ \& integrate them into in-house simulation workflows.
- Streamline molecular simulations \& maximize resource utilization via Slurm across HPC clusters of 50+ TB simulation datasets, earning NSF ACCESS grant.
- Mentor 11 researchers on Git, HPC scripting, \& simulation methods, earning APS Career Mentor Fellowship.
- Present award-winning research at 17 conferences, including GRC 2024 \& USF 2023, on rubber dynamics \& copolymer design.
#
- Architected C++/Python simulation tools for polymer deformation and thermal stability, enabling fast iteration across copolymer and rubber design space.
- Optimized high-throughput workflows to process 50+ TB datasets on HPC clusters, reducing runtime by 90\% and earning NSF ACCESS grant.
- Profiled and debugged large C++ codebases using GDB and valgrind to resolve memory leaks, segmentation faults, and scaling inefficiencies.
- Applied Bayesian optimization to model dynamic behavior \& extract glass transitions.
- Mentored 15+ researchers on simulation, profiling, and scientific computing, fostering collaboration and increasing group velocity.
- Led internal documentation and technical presentations to standardize workflows and sustain long-term code maintainability.
