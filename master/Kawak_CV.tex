%%%%%%%%%%%%%%%%%%%%%%%%%%%%%%%%%%%%%%%%%%%%%%%%%%%%%%%%%%%%%%%%%%%%%%%%
% TAILORING GUIDE FOR FUTURE CHATGPT CLIENTS
%
% Purpose:
% This master CV provides modular content with multiple specialization options.
% Follow the instructions below to generate a position-specific CV.
%
% Tailoring Rules:
% 1. Review the tailoring configurations below to select relevant content.
% 2. After selection, REMOVE the corresponding subsection titles such as:
%       \textbf{Material Simulation and Analysis}
%       \textbf{Software Development and Automation}
%    Retain only the higher-level section titles, such as:
%       \textbf{Postdoctoral Researcher}, University of South Florida \hfill \textit{2022 – Present} \\
%
% Suggested Tailoring Configurations:
%
% 1. Software R&D / Scientific Computing Positions
%    - Keep:
%      * Professional Summary
%      * Technical Skills
%      * Software Development and Automation (Postdoc)
%      * High-Performance Computing Optimization (Postdoc)
%      * Debugging, Profiling, and Performance Tuning (Postdoc)
%      * Software Development and GPU Acceleration (PhD)
%      * HPC and Workflow Optimization (PhD)
%    - Optional:
%      * Scientific Communication and Collaboration (PhD)
%    - Remove:
%      * Materials Simulation and Analysis (Postdoc)
%      * Structure–Property Analysis (PhD)
%
% 2. Materials Scientist / Simulation Researcher Positions
%    - Keep:
%      * Professional Summary
%      * Technical Skills
%      * Material Simulation and Analysis (Postdoc)
%      * Structure–Property Analysis (PhD)
%      * HPC Optimization and Workflow Acceleration (Postdoc)
%      * Scientific Communication and Collaboration (PhD)
%    - Optional:
%      * Software Development and Automation (Postdoc)
%    - Remove:
%      * Debugging, Profiling, and Performance Tuning (Postdoc)
%      * Software Development and GPU Acceleration (PhD)
%
% 3. Multidisciplinary Leadership / Team Management Roles
%    - Keep:
%      * Professional Summary
%      * Technical Skills
%      * Leadership, Mentorship, and Community Engagement (Postdoc)
%      * Scientific Communication and Collaboration (PhD)
%      * Software and HPC sections (if leadership is technical)
%    - Optional:
%      * All technical sections based on role expectations
%    - Remove:
%      * None, unless character limit requires further trimming
%%%%%%%%%%%%%%%%%%%%%%%%%%%%%%%%%%%%%%%%%%%%%%%%%%%%%%%%%%%%%%%%%%%%%%%%
\documentclass[letterpaper,12pt]{article}
\usepackage[margin=1.00in, top=0.7in, headsep=20pt]{geometry}
\usepackage{titlesec}
\usepackage{enumitem}
\usepackage{hyperref}
\usepackage{parskip}
\usepackage{fontawesome}

\hypersetup{
  colorlinks=true,
  urlcolor=blue,
}

% Font
\usepackage[T1]{fontenc}
\usepackage[sc]{mathpazo}

\usepackage[backend=biber,sorting=mysort,style=numeric-comp,defernumbers,maxbibnames=20]{biblatex}
\addbibresource[label=articles]{articles.bib}
\titleformat{\section}{\large\bfseries}{}{0em}{}[\titlerule]
 % fix header
 \defbibheading{fix}[\refname]{\section*{#1}}
 

\AtBeginBibliography{\small}

  % This is the sorting scheme for the CV
  \DeclareSortingScheme{mysort}{
    \sort[direction=descending]{
      \field{year}
    }
    \sort[direction=descending]{
      \field[padside=left,padwidth=2,padchar=0]{month}
      \literal{00}
    }
  }

  % Use this for reversing the numbers (count down) in the bibliography
  % Count total number of entries in each refsection
  \AtDataInput{%
  \csnumgdef{entrycount:\therefsection}{%
  \csuse{entrycount:\therefsection}+1}}

  % Print the labelnumber as the total number of entries in the
  % current refsection, minus the actual labelnumber, plus one
  \DeclareFieldFormat{labelnumber}{\mkbibdesc{#1}}    
  \newrobustcmd*{\mkbibdesc}[1]{%
  \number\numexpr\csuse{entrycount:\therefsection}+1-#1\relax}

% Custom sections
\usepackage[]{titlesec}
\titleformat{\section}{
  \rmfamily\mdseries\Large\bfseries
  }{}{}{}[\vspace{0.1ex}\titlerule]
\titlespacing*{\section}{0ex}{3ex}{0.8ex}
\titlespacing*{\therefsection}{0ex}{3ex}{0.8ex}
\titlespacing*{\refsection}{0ex}{3ex}{0.8ex}

\newlist{tabitemize}{itemize}{1}
%\setlist[tabitemize]{label=\textbullet,nosep,align=parleft,leftmargin=*,}
\setlist[tabitemize]{label=\textbullet,itemsep=1pt,align=parleft,leftmargin=*,}

%% Customize page headers
%\pagestyle{myheadings}
%\markright{\name}
%\thispagestyle{empty}
\usepackage[]{fancyhdr}
\pagestyle{fancy}
\fancyhead{}
\renewcommand{\headrulewidth}{0pt}
%\fancyhead[RO,LE]{Pierre Kawak, Ph.D.}
\fancyfoot{}
\fancyfoot[LO,CE]{\thepage}
\fancyfoot[RO,LE]{Pierre Kawak, Ph.D.}
\begin{document}

\begin{center}
  {\LARGE \textbf{Pierre Kawak, Ph.D.}}\\
  \faPhone\ +1 (801) 762-7999 \quad \faEnvelope\ \href{mailto:pskawak@gmail.com}{pskawak@gmail.com} \quad \faLink\ \href{https://linktr.ee/pkawak}{linktr.ee/pkawak}
\end{center}

\vspace{-0.3\baselineskip}
\section*{Professional Summary}
Computational polymer physicist with expertise in molecular simulation, statistical mechanics, and polymer theory. Extensive experience leading collaborative projects at the intersection of materials science, chemical engineering, and polymer physics.
\vspace{0.5em}
\begin{tabitemize}
    \item Expert in developing, optimizing, and deploying molecular dynamics and Monte Carlo simulations to uncover structure–property relationships in polymers and soft materials.
    \item Experienced in leading cross-functional teams, managing up to 11 researchers, and mentoring graduate and undergraduate students on computational methods and scientific communication.
    \item Proven ability to architect and accelerate analysis pipelines for polymer simulations, including profiling and optimizing in-house codes in Python, C++, and CUDA.
    \item Demonstrated success translating molecular insights into design principles for tough, recyclable, and sustainable materials, including elastomeric nanocomposites and copolymer systems.
    \item Strong record of scientific communication with multiple invited talks, publications in top journals, and community leadership in the polymer physics community.
    \item Passionate about building inclusive scientific communities, with leadership experience organizing professional development events, symposia, and mentoring programs.
\end{tabitemize}

\vspace{-0.3\baselineskip}
\section*{Technical Skills}
\begin{tabitemize}
    \item \textbf{Simulation and Modeling:} Molecular dynamics (MD), Monte Carlo (MC), coarse-grained modeling, polymer chain modeling, non-equilibrium simulations, phase behavior analysis, free energy calculations.
    \item \textbf{Software Development:} Architecting, profiling, and deploying scientific software in Python, C++, CUDA, Bash, and FORTRAN; experience with collaborative software development using Git.
    \item \textbf{Simulation Packages:} LAMMPS, HOOMD-blue, GROMACS, OpenMM, Lattice-based MC codes, in-house simulation and analysis tools.
    \item \textbf{Analysis and Data Processing:} Custom analysis pipelines for molecular simulation data using Python (NumPy, SciPy, Matplotlib, Pandas), bash scripting, and parallel computation.
    \item \textbf{High-Performance Computing:} Experience with HPC job scheduling (SLURM), running simulations on national supercomputing facilities (XSEDE, ACCESS), and GPU-accelerated computing (CUDA, Nsight, nvprof).
    \item \textbf{Data Visualization and Reporting:} Scientific visualization with Matplotlib, Plotly, and VMD; automated report generation in LaTeX.
    \item \textbf{Process Simulation and Thermodynamics:} Experience with process simulation tools (Aspen HYSYS) and instruction in thermodynamic modeling.
    \item \textbf{Leadership and Project Management:} Leading cross-functional teams, mentoring junior researchers, organizing professional development workshops and symposia.
\end{tabitemize}


\vspace{-0.3\baselineskip}
\section*{Research Experience}

\textbf{Postdoctoral Researcher}, University of South Florida \hfill \textit{2022 – Present} \\
\textit{Advisor: Prof. David Simmons}
\vspace{0.5em}

\textbf{Material Simulation and Analysis}
\begin{tabitemize}
    \item Developed molecular dynamics models (LAMMPS, GROMACS, OPLS) to investigate stress relaxation, deformation, and nanoscale reinforcement in polymer nanocomposites.
    \item Simulated thermo-elastoplastic deformation and rheology of rubber and copolymers, identifying nanoscale toughening mechanisms.
    \item Modeled copolymer sequences with enhanced glass transition temperatures without altering processing or feedstock chemistry.
    \item Applied Bayesian optimization to fit relaxation dynamics and extract glass transition behavior.
    \item Led multi-dimensional parameter sweeps (nanoparticle size, chemistry, volume fraction) to optimize nanocomposite and copolymer designs.
    \item Collaborated with experimentalists to validate molecular predictions against thermal and mechanical performance targets.
    \item Designed statistical models and applied machine learning to extract insights from large molecular datasets.
\end{tabitemize}

\vspace{0.5em}
\textbf{Software Development and Automation}
\begin{tabitemize}
    \item Architected Python/C++/CUDA simulation and analysis tools to enable high-throughput, physics-informed property prediction.
    \item Developed modular, reusable rheology and stress analysis toolkits integrated into team-wide simulation workflows.
    \item Automated data analysis pipelines handling 50+ TB datasets, reducing turnaround time by 90\%.
    \item Built bash, Slurm, and MPI automation for large-scale HPC job orchestration, checkpointing, and storage.
    \item Documented group codebases and created tutorials for reproducible scientific computing.
    \item Integrated local stress analysis tools to study mechanical response under deformation.
\end{tabitemize}

\vspace{0.5em}
\textbf{High-Performance Computing Optimization and Workflow Acceleration}
\begin{tabitemize}
    \item Led optimization of 50+ TB data pipelines across HPC clusters, improving data throughput by 90\%.
    \item Streamlined multi-node CPU and GPU workflows for molecular simulations.
    \item Reduced simulation runtime by 90\% through parallelization and high-performance computing resource optimization.
    \item Awarded NSF ACCESS Compute Resource Grant for advancing scalable HPC-enabled molecular simulations.
\end{tabitemize}

\vspace{0.5em}
\textbf{Debugging, Profiling, and Performance Tuning}
\begin{tabitemize}
    \item Debugged and profiled C++/Python group codebases using GDB, valgrind, and profiling tools (nsight, nvprof), resolving memory leaks and runtime bottlenecks.
    \item Improved software modularity, runtime, and scalability for multi-month, high-dimensional simulations.
    \item Led efforts to standardize code maintainability and scientific reproducibility through internal documentation and technical presentations.
\end{tabitemize}

\vspace{0.5em}
\textbf{Leadership, Mentorship, and Community Engagement}
\begin{tabitemize}
    \item Led and mentored 11+ researchers in HPC, molecular simulation, version control, and scientific computing practices.
    \item Designed structured onboarding programs, documentation, and tutorials to train new users.
    \item Organized professional development workshops, DEI programs, and research symposia for 200+ researchers across multiple organizations.
    \item Presented research at 17+ institutional, industrial, and academic conferences, earning multiple awards (GRC 2024, USF 2023).
    \item Recognized with the APS Career Mentor Fellowship for leadership in mentoring and community building.
\end{tabitemize}

\textbf{Ph.D. Researcher}, Brigham Young University \hfill \textit{2017 – 2022} \\
\textit{Advisor: Prof. Douglas Tree}

\vspace{0.5em}
\textbf{Software Development and GPU Acceleration}
\begin{tabitemize}
    \item Built two GPU-accelerated Monte Carlo engines in C++/CUDA to simulate polymer crystallization and free energy landscapes.
    \item Developed advanced sampling algorithms (Configurational Bias Monte Carlo, Wang-Landau) to efficiently explore high-dimensional molecular states.
    \item Designed GPU-optimized data structures (cell lists) and pairwise interaction search algorithms, achieving significant performance improvements.
    \item Utilized NVIDIA profiling tools (nsight, nvprof) to debug, optimize, and scale GPU-accelerated codes.
\end{tabitemize}

\vspace{0.5em}
\textbf{Structure–Property Analysis and Data Processing}
\begin{tabitemize}
    \item Developed numerical algorithms to quantify nanoscale crystalline and orientational order from 3D simulation trajectories.
    \item Processed large-scale simulation data to generate thermodynamic observables, including free energy derivatives and heat capacity profiles.
    \item Extracted phase diagrams and structure–property relationships through systematic parameter sweeps and automated analysis pipelines.
\end{tabitemize}

\vspace{0.5em}
\textbf{High-Performance Computing and Workflow Optimization}
\begin{tabitemize}
    \item Leveraged HPC resources to perform long-timescale, high-throughput Monte Carlo simulations using GPU acceleration.
    \item Automated simulation and data analysis workflows to enable reproducible exploration of high-dimensional parameter spaces.
\end{tabitemize}

\vspace{0.5em}
\textbf{Scientific Communication and Collaboration}
\begin{tabitemize}
    \item Collaborated with interdisciplinary teams to apply simulation methods to polymer crystallization and glass transition problems.
    \item Presented research findings at academic conferences, advancing understanding of nanoscale phenomena in polymer materials.
\end{tabitemize}

\textbf{Graduate Researcher}, American University of Sharjah \hfill \textit{2015 – 2017} \\
\textit{Advisor: Prof. Ghaleb Husseini}
\begin{tabitemize}[leftmargin=*]
  \item Synthesized estrone-functionalized drug nanocarriers; enhanced release control for chemotherapy applications.
  \item Validated drug stability \& kinetics with DLS/NMR; optimized ultrasonic parameters.
  \item Standardized lab protocols; boosted reproducibility and cross-lab collaboration.
  \item Presented at 3 conferences; awarded Best Talk at AUS Biomedical Symposium.
\end{tabitemize}

\vspace{-0.3\baselineskip}
\section*{Leadership \& Community Engagement}
\begin{tabitemize}[leftmargin=*]
  \item \textbf{President}, Early Career Researchers in Polymer Physics (2022–Present): Led 550+ member global network, organized 150+ attendee virtual symposium.
  \item \textbf{Founder \& President}, USF Postdoctoral Scholar Association (2023–Present): Launched NPA-funded ELEVATE Talk Series and DEI programs for 200+ postdocs.
  \item \textbf{Founder \& President}, BYU Chem.~Eng.~Graduate Council (2019–2022): Shaped department policies and spearheaded outreach and recruitment.
\end{tabitemize}

%%%%%%%%%%%%%%%%%%%%%%%%%%%%%%%%
%% Articles and Presentations %%
%%%%%%%%%%%%%%%%%%%%%%%%%%%%%%%%
%\vspace{-1.5\baselineskip}
\vspace{-0.3\baselineskip}
%recompile from start if numbers aren't right
\begin{refsection}[articles]
  \nocite{*}
  \setlength\bibitemsep{0pt}
  \printbibliography[resetnumbers=true,type=article,title={Selected Peer-Reviewed Publications},heading=fix]
\end{refsection}

\vspace{-0.3\baselineskip}
\section*{Education}
\textbf{Ph.D.} in Chemical Engineering, Brigham Young University \hfill \textit{2022} \\
\textbf{M.S.} in Chemical Engineering, American University of Sharjah \hfill \textit{2017} \\
\textbf{B.S.} in Chemical Engineering (Econ. Minor), American University of Sharjah \hfill \textit{2015}

%\vspace{-0.3\baselineskip}
\vspace{0.5em}
\noindent\textit{Full list of publications and presentations available at \href{https://linktr.ee/pkawak}{linktr.ee/pkawak}}

\end{document}
