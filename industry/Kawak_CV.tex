\documentclass[letterpaper,12pt]{article}
\usepackage[margin=1.00in, top=0.7in, headsep=20pt]{geometry}
\usepackage{titlesec}
\usepackage{enumitem}
\usepackage{hyperref}
\usepackage{parskip}
\usepackage{fontawesome}

\hypersetup{
  colorlinks=true,
  urlcolor=blue,
}

% Font
\usepackage[T1]{fontenc}
\usepackage[sc]{mathpazo}

\usepackage[backend=biber,sorting=mysort,style=numeric-comp,defernumbers,maxbibnames=20]{biblatex}
\addbibresource[label=articles]{articles.bib}
\titleformat{\section}{\large\bfseries}{}{0em}{}[\titlerule]
 % fix header
 \defbibheading{fix}[\refname]{\section*{#1}}
 

\AtBeginBibliography{\small}

  % This is the sorting scheme for the CV
  \DeclareSortingScheme{mysort}{
    \sort[direction=descending]{
      \field{year}
    }
    \sort[direction=descending]{
      \field[padside=left,padwidth=2,padchar=0]{month}
      \literal{00}
    }
  }

  % Use this for reversing the numbers (count down) in the bibliography
  % Count total number of entries in each refsection
  \AtDataInput{%
  \csnumgdef{entrycount:\therefsection}{%
  \csuse{entrycount:\therefsection}+1}}

  % Print the labelnumber as the total number of entries in the
  % current refsection, minus the actual labelnumber, plus one
  \DeclareFieldFormat{labelnumber}{\mkbibdesc{#1}}    
  \newrobustcmd*{\mkbibdesc}[1]{%
  \number\numexpr\csuse{entrycount:\therefsection}+1-#1\relax}

% Custom sections
\usepackage[]{titlesec}
\titleformat{\section}{
  \rmfamily\mdseries\Large\bfseries
  }{}{}{}[\vspace{0.1ex}\titlerule]
\titlespacing*{\section}{0ex}{3ex}{0.8ex}
\titlespacing*{\therefsection}{0ex}{3ex}{0.8ex}
\titlespacing*{\refsection}{0ex}{3ex}{0.8ex}

\newlist{tabitemize}{itemize}{1}
%\setlist[tabitemize]{label=\textbullet,nosep,align=parleft,leftmargin=*,}
\setlist[tabitemize]{label=\textbullet,itemsep=1pt,align=parleft,leftmargin=*,}

%% Customize page headers
%\pagestyle{myheadings}
%\markright{\name}
%\thispagestyle{empty}
\usepackage[]{fancyhdr}
\pagestyle{fancy}
\fancyhead{}
\renewcommand{\headrulewidth}{0pt}
%\fancyhead[RO,LE]{Pierre Kawak, Ph.D.}
\fancyfoot{}
\fancyfoot[LO,CE]{\thepage}
\fancyfoot[RO,LE]{Pierre Kawak, Ph.D.}
\begin{document}

\begin{center}
  {\LARGE \textbf{Pierre Kawak, Ph.D.}}\\
  \faPhone\ +1 (801) 762-7999 \quad \faEnvelope\ \href{mailto:pskawak@gmail.com}{pskawak@gmail.com} \quad \faLink\ \href{https://linktr.ee/pkawak}{linktr.ee/pkawak}
\end{center}

\vspace{-0.3\baselineskip}
\section*{Professional Summary}
Computational polymer scientist with 11+ years advancing materials and drug delivery through high-impact modeling, simulation, and cross-functional collaboration — delivering scalable tools, 50+ TB HPC workflows, and award-winning insights across academia and industry. Passionate about accelerating discovery at the intersection of polymers, computation, and applied research.

\vspace{-0.3\baselineskip}
\section*{Technical Skills}
\begin{tabitemize}[leftmargin=*]
  \item \textbf{Programming \& Automation:} Python, C++, CUDA, MATLAB, Bash, R, Julia
  \item \textbf{Simulation \& Modeling:} LAMMPS, GROMACS, Gaussian, AMBER, OPLS, Molecular Dynamics, Monte Carlo, Free Energy Calculations
  \item \textbf{Analysis \& Visualization:} VMD, OVITO, NumPy, Pandas, scikit-learn, Matplotlib
  \item \textbf{High-Performance Computing:} Slurm, MPI, Parallelism, Automation, 50TB+ Data
  \item \textbf{Experimental Techniques:} Drug Encapsulation, DLS, NMR, Liposomal Formulations
  \item \textbf{Communication \& Leadership:} Scientific Writing, Mentorship, Advocacy, 27+ Conference Talks, 5 Publications
\end{tabitemize}

\vspace{-0.3\baselineskip}
\section*{Research Experience}

\textbf{Postdoctoral Researcher}, University of South Florida \hfill \textit{2022 – Present} \\
\textit{Advisor: Prof. David Simmons}
\begin{tabitemize}[leftmargin=*]
  \item Simulated polymer deformation to inform composite design strategies at the nanoscale.
  \item Enhanced copolymer glass transition temperature via sequence-specific simulations; improving thermal stability without altering feedstock or processing.
  \item Developed rheology tools and extended internal analysis codebase; accelerating workflows and boosting team efficiency.
  \item Streamlined HPC pipelines processing 50+ TB datasets; cut analysis time by 90\% and earned NSF ACCESS grant.
  \item Mentored 11 researchers in simulations, Git, and HPC; named APS Mentoring Fellow.
  \item Delivered 17 conference talks; earned awards at GRC (2024) and USF Symp.~(2023).
\end{tabitemize}

\textbf{Ph.D. Researcher}, Brigham Young University \hfill \textit{2017 – 2022} \\
\textit{Advisor: Prof. Douglas Tree}
\begin{tabitemize}[leftmargin=*]
  \item Developed two GPU-accelerated Monte Carlo simulation codes in C++/CUDA; accelerated crystallization research by 100×.
  \item Generated 3D free energy landscapes \& phase diagrams using novel order parameters.
  \item Analyzed large 3D datasets with OVITO to uncover kinetic and structural transitions.
  \item Mentored 4 undergraduates; co-authored 2 papers and 6 conference abstracts.%; supported grad school admissions.
  \item Won APS Distinguished Student Award and BYU Research Presentation Award.
  \item Contributed key data to a successful \$500K NSF CAREER proposal.
\end{tabitemize}

\textbf{Graduate Researcher}, American University of Sharjah \hfill \textit{2015 – 2017} \\
\textit{Advisor: Prof. Ghaleb Husseini}
\begin{tabitemize}[leftmargin=*]
  \item Synthesized estrone-functionalized drug nanocarriers; enhanced release control for chemotherapy applications.
  \item Validated drug stability \& kinetics with DLS/NMR; optimized ultrasonic parameters.
  \item Standardized lab protocols; boosted reproducibility and cross-lab collaboration.
  \item Presented at 3 conferences; awarded Best Talk at AUS Biomedical Symposium.
\end{tabitemize}

\vspace{-0.3\baselineskip}
\section*{Leadership \& Community Engagement}
\begin{tabitemize}[leftmargin=*]
  \item \textbf{President}, Early Career Researchers in Polymer Physics (2022–Present): Led 550+ member global network, organized 150+ attendee virtual symposium.
  \item \textbf{Founder \& President}, USF Postdoctoral Scholar Association (2023–Present): Launched NPA-funded ELEVATE Talk Series and DEI programs for 200+ postdocs.
  \item \textbf{Founder \& President}, BYU Chem.~Eng.~Graduate Council (2019–2022): Shaped department policies and spearheaded outreach and recruitment.
\end{tabitemize}

%%%%%%%%%%%%%%%%%%%%%%%%%%%%%%%%
%% Articles and Presentations %%
%%%%%%%%%%%%%%%%%%%%%%%%%%%%%%%%
%\vspace{-1.5\baselineskip}
\vspace{-0.3\baselineskip}
%recompile from start if numbers aren't right
\begin{refsection}[articles]
  \nocite{*}
  \setlength\bibitemsep{0pt}
  \printbibliography[resetnumbers=true,type=article,title={Selected Peer-Reviewed Publications},heading=fix]
\end{refsection}

\vspace{-0.3\baselineskip}
\section*{Education}
\textbf{Ph.D.} in Chemical Engineering, Brigham Young University \hfill \textit{2022} \\
\textbf{M.S.} in Chemical Engineering, American University of Sharjah \hfill \textit{2017} \\
\textbf{B.S.} in Chemical Engineering (Econ. Minor), American University of Sharjah \hfill \textit{2015}

%\vspace{-0.3\baselineskip}
\vspace{0.5em}
\noindent\textit{Full list of publications and presentations available at \href{https://linktr.ee/pkawak}{linktr.ee/pkawak}}

\end{document}
